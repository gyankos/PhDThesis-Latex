\documentclass[tikz]{standalone}
\usepackage{eulervm}
\usepackage[osf,sc]{mathpazo}
\usepackage{inconsolata}
\usetikzlibrary{
  % https://github.com/Qrrbrbirlbel/pgf/blob/master/tikzlibrarypositioning-plus.code.tex
  positioning-plus,
  % https://github.com/Qrrbrbirlbel/pgf/blob/master/tikzlibrarynode-families.code.tex
  node-families,
  chains,backgrounds}
\begin{document}
\begin{tikzpicture}[
  >=latex, thick,
  node distance=1cm and 1.8cm,
  my edge/.style={->}, my edge'/.style={<-},
  box/.style={
    shape=rectangle, draw, fill, font=\strut,
    rounded corners},
  wh box/.style={box, fill=white, Minimum Height=all,inner sep=0pt, text width=24mm,align=center},
  bl box/.style={box, text=white, align=center, text width=1.8cm, Minimum Height=all},
  gr box/.style={box, thick, inner sep=+.5cm, draw=gray, fill=gray!25}
]


%% Products
\begin{scope}[start chain=1 going below,nodes={wh box, Minimum Width=d}]
\foreach \Text[count=\cnt from 1] in {Shiny, Brighty, CleanHand, Marseille, Milk, Yogurt, Water, Coffee}
\node[on chain=1] (p\cnt) {\Text};
\end{scope}
\node[above=of p1] (p) {\Large \texttt{Product}};

%% Categories
\begin{scope}[nodes={wh box, Minimum Width=d}]
\foreach \Text[count=\cnt from 1,evaluate=\cnt as \cnti using \cnt+2] in {Cleaner, Soap, Diary Product, Drink}
\node[left=44 mm of p\cnti] (c\cnt) {\Text};
\end{scope}
\node[left=24 mm of p] (c) {\Large \texttt{Category}};
\node[left=24 mm of p8] (c8) {};

%% Types
\node[wh box, Minimum Width=d,left=22 mm of c2] (t1) {House Cleaner};
\node[wh box, Minimum Width=d,left=22 mm of c3] (t2) {Food};
\node[left=24 mm of c] (t) {\Large \texttt{Type}};
\node[left=44 mm of c8] (t8) {};


\begin{scope}[on background layer, nodes=gr box]
  \node[gr box,fit=(p)(p8)] {};
  \node[gr box,fit=(c)(c8)] {};
  \node[gr box,fit=(t)(t1)(t8)] {};
\end{scope}




\draw[my edge'] (t1) -- (c1);
\draw[my edge'] (c1) -- (p1);
\draw[my edge'] (c1) -- (p2);
\draw[my edge'] (t1) -- (c2);
\draw[my edge'] (c2) -- (p3);
\draw[my edge'] (c2) -- (p4);
\draw[my edge'] (t2) -- (c3);
\draw[my edge'] (c3) -- (p5);
\draw[my edge'] (c3) -- (p6);
\draw[my edge'] (t2) -- (c4);
\draw[my edge'] (c4) -- (p7);
\draw[my edge'] (c4) -- (p8);

\end{tikzpicture}
\end{document}
