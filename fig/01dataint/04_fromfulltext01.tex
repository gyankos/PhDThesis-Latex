\documentclass[tikz,4pt]{standalone}

\usetikzlibrary{arrows,shapes,positioning}
\usetikzlibrary{calc,decorations.markings}

\usepackage{varwidth}
\usepackage{relsize}

\tikzset{fontscale/.style = {font=\relsize{#1}}}
\tikzset{font={\fontsize{8pt}{12}\selectfont}}

% taken from manual
\makeatletter
\pgfdeclareshape{document}{
\inheritsavedanchors[from=rectangle] % this is nearly a rectangle
\inheritanchorborder[from=rectangle]
\inheritanchor[from=rectangle]{center}
\inheritanchor[from=rectangle]{north}
\inheritanchor[from=rectangle]{south}
\inheritanchor[from=rectangle]{west}
\inheritanchor[from=rectangle]{east}
% ... and possibly more
\backgroundpath{% this is new
% store lower right in xa/ya and upper right in xb/yb
\southwest \pgf@xa=\pgf@x \pgf@ya=\pgf@y
\northeast \pgf@xb=\pgf@x \pgf@yb=\pgf@y
% compute corner of ‘‘flipped page’’
\pgf@xc=\pgf@xb \advance\pgf@xc by-10pt % this should be a parameter
\pgf@yc=\pgf@yb \advance\pgf@yc by-10pt
% construct main path
\pgfpathmoveto{\pgfpoint{\pgf@xa}{\pgf@ya}}
\pgfpathlineto{\pgfpoint{\pgf@xa}{\pgf@yb}}
\pgfpathlineto{\pgfpoint{\pgf@xc}{\pgf@yb}}
\pgfpathlineto{\pgfpoint{\pgf@xb}{\pgf@yc}}
\pgfpathlineto{\pgfpoint{\pgf@xb}{\pgf@ya}}
\pgfpathclose
% add little corner
\pgfpathmoveto{\pgfpoint{\pgf@xc}{\pgf@yb}}
\pgfpathlineto{\pgfpoint{\pgf@xc}{\pgf@yc}}
\pgfpathlineto{\pgfpoint{\pgf@xb}{\pgf@yc}}
\pgfpathlineto{\pgfpoint{\pgf@xc}{\pgf@yc}}
}
}
\makeatother


% \begin{tikzpicture}
%   \node[doc,align=left] (x) {{\large \textbf{Balance disorder}}\\
%   \dots The symptoms may be recurring \\
%   or relatively constant. When \\
%   symptoms exist, they may include: \\
%   a sensation of dizziness or vertigo,\\
%   lightheadedness or feeling woozy, \\
%   problems reading and difficulty \\
%   seeing or disorientation \dots
% };
%   \node[doc,align=left] at (7,0) (y)
%   {\dots\\
%   \textbf{Vertigo}: \textit{780.4}\\
%     $\to$ M\'eni\`ere's disease, \textit{386.00}\\
%     $\to$ Vestibular, \textit{386.10}\\
%     $\to$ Cerebral dysfunction, \textit{386.2}\\
%     $\to$ Peripheral, \textit{386.10}\\
%     $\to$ \dots\\
%     };
%
% \node[doc,align=left] at (0,-6) {\dots\\
% \textbf{General symptoms}: \textit{780}\\
%   $\to$ Alteration of consciousness, \textit{780.0}\\
%   $\to$ Hallucinations, \textit{780.1}\\
%   $\to$ Syncope and collapse, \textit{780.2}\\
%   $\to$ Dizziness and giddiness, \textit{780.4}\\
%   $\to$ \dots\\};
%
% \node[doc] at (7,-6) {\includegraphics[width=\textwidth]{tumorSalivary.jpg}};

\begin{document}

\tikzstyle{doc}=[%
draw,
thick,
align=center,
color=black,
shape=document,
minimum height=56.4mm,
minimum width=40mm,
shape=document,
text centered,
execute at begin node={\begin{varwidth}{40mm}},
execute at end node={\end{varwidth}},
align=left
]

\begin{tikzpicture}

  \node[doc,align=left] (y)
  {\dots\\
  \textbf{Vertigo}: \textit{780.4}\\
    $\to$ M\'eni\`ere's disease, \textit{386.00}\\
    $\to$ Vestibular, \textit{386.10}\\
    $\to$ Cerebral dysfunction, \textit{386.2}\\
    $\to$ Peripheral, \textit{386.10}\\
    $\to$ \dots\\
    };

\node[above right=of y] (w) {};
\node[above=of w] (v) {Vertigo};
\node[right=of v] (vi) {780.4};
\draw[<->] (v) edge node [above,midway,sloped] {hasCode} (vi);

\node[below right=of v] (m) {M\'eni\`ere's disease};
\node[right=of m] (mi) {386.00};
\draw[->] (v) edge[bend right] node [above,midway,sloped] {symptomOf}  (m);
\draw[<->] (m) edge node [above,midway,sloped] {hasCode} (mi);

\node[below=of m] (e) {Vestibular};
\node[right=of e] (ei) {386.10};
\draw[->] (v) edge[bend right] node [above,midway,sloped] {canBe}  (e);
\draw[<->] (e) edge node [above,midway,sloped] {hasCode} (ei);

%     $\to$ Peripheral, \textit{386.10}\\
\node[below=of e] (p) {Peripheral};
\draw[->] (v) edge[bend right]  node [above,midway,sloped] {canBe}  (p);
\draw[<->] (p) edge[bend right] node [above,midway,sloped] {hasCode} (ei);

\node[below=of p] (c) {Cerebral dysfunction};
\node[right=of c] (ci) {386.2};
\draw[->] (v) edge[bend right] node [above,midway,sloped] {symptomOf}  (c);
\draw[<->] (c) edge node [above,midway,sloped] {hasCode} (ci);



\end{tikzpicture}

\end{document}
