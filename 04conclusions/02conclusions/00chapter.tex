\chapter{Conclusions}\label{cha:IteMissaEst}
\epigraph{``Cuius rei demonstrationem mirabilem sane detexi hanc marginis exiguitas non caperet.''}{--- Pierre de Fermat}
%\epigraph{\textit{My friends, my friends\\ Let us not forget this heritage\\ That our parents have left us\\ Let us keep it fondly\\ This heritage is our identity}.}{--- Bombino, Imidiwan (en. `\textit{My friends}')}

\textit{We remind our reader that each chapter provided a conclusion section, where we provided the intended future works for each thesis topic.}

This thesis introduced the \textsc{Generalized Semistructured Model}, a data model allowing the representation of both graphs and nested data.  This has led to the definition of an intermediate data model, \textbf{nested graphs}, which allows both to have a data structure with two main object classes (vertices and edges), and the data to be nested. However, this thesis has not treated other data models, such as RDF, where there are at least three classes of objects (note that some RDF properties may act as both vertices and edges). This requires the definition of ad hoc operators for this additional data model. Similar considerations may be adopted for hypergraphs, too. Therefore, we leave the definition of these operators  to future developments in our research through the usage of \textbf{GSQL} over GSM data. % which, as we have already shown with many examples, allows to represent different types of interrogation languages GSM data.

In particular, GSQL supported the definition of both \textbf{graph joins} and \textbf{graph nestings} that are required operations within the context of data integration: while the former operation allowed to chain the matching vertices and creates new edges by using an user-defined \textbf{es} semantics, the latter operation is required in the ``blocking'' operations where \begin{mylist}
	\item the original obects' pieces of information are preserved and when
	\item we may reduce the data representation at its coarsest representation level.
\end{mylist}
We also saw that the implementation of both graph operators revealed the deficiencies of current graph query languages in providing those operations, either because of their data model or because of their query plan. These limitations demand for a new query language allowing these two (nested) graph operations in an efficient implementation. 

Last, we showed that  GSM allowed the data integration between different data representations and GSQL may be used to express both data integration and data mining tasks, thus showing that such naïve query language may be used like an assembly interface towards which we can express different possible query languages. Further work has still to be carried out at the GSQL optimization level: equivalence axioms and new aggregated operators allowing computational enhancement frequently occurring in data querying tasks (similar to the composition $\bowtie$ and $\sigma_\theta$ providing the $\bowtie_\theta$ operator) are still to be provided. Moreover, most interesting graph features may arise whether GSQL may deal with graph data uncertainty \cite{Getoor07} and graph metrics \cite{DMR}: further work should also analyse the ability of such language to manipulate data alongside with uncertainty measures.