\documentclass[a4paper,UKenglish,part]{lipicsmaster}

%\usepackage{draftwatermark}

\newcommand{\zero}{\mathfrak{0}}

\newcommand{\hmmax}{100}
\newcommand{\bmmax}{100}
\usepackage{bbm}
\usepackage{amsfonts}
\usepackage{makeidx}
\makeindex

%% Mathematical fonts
\DeclareSymbolFont{extraitalic}      {U}{zavm}{m}{it}
\DeclareMathSymbol{\Qoppa}{\mathord}{extraitalic}{161}
\DeclareMathSymbol{\qoppa}{\mathord}{extraitalic}{162}
\DeclareMathSymbol{\Stigma}{\mathord}{extraitalic}{167}
\DeclareMathSymbol{\Sampi}{\mathord}{extraitalic}{165}
\DeclareMathSymbol{\sampi}{\mathord}{extraitalic}{166}
\DeclareMathSymbol{\stigma}{\mathord}{extraitalic}{168}
\DeclareMathOperator{\dom}{dom}
\DeclareMathOperator{\cod}{cod}
%\DeclareMathOperator*{\argmin}{arg\,min} % thin space, limits underneath in displays
%\DeclareMathOperator*{\argmax}{arg\,max} % thin space, limits underneath in displays
%\DeclareMathOperator*{\argmax}{argmax} % no space, limits underneath in displays
\DeclareMathOperator{\argmax}{arg\,max} % thin space, limits on side in displays
\DeclareMathOperator{\argmin}{arg\,min} % thin space, limits on side in displays
%\DeclareMathOperator{\argmax}{argmax} % no space, limits on side in displays
\newcommand{\nat}{\mathbb{N}}
\newcommand{\partof}[1]{\wp(#1)}
\newcommand{\predset}[1]{\mathcal{P}(#1)}
\newcommand{\eqdef}{\overset{\mathrm{def}}{=\mathrel{\mkern-3mu}=}}

\usepackage{microtype,adjustbox}

  %for A4 paper format use option "a4paper", for US-letter use option "letterpaper"
  %for british hyphenation rules use option "UKenglish", for american hyphenation rules use option "USenglish"
  \usepackage[utf8]{inputenc}
\usepackage{epigraph}
\usepackage{epipart}
\usepackage{microtype}%if unwanted, comment out or use option "draft"
 % add possibly `sc` and `osf` options
\usepackage{eulervm}
\usepackage[osf,sc]{mathpazo}

%\DeclareSymbolFont{greekletters}{OML}{cmr}{m}{it}
%\DeclareMathSymbol{\varrho}{\mathalpha}{greekletters}{"25}
%\DeclareMathSymbol{\varsigma}{\mathalpha}{greekletters}{"26}
%\DeclareMathSymbol{\varphi} {\mathalpha}{letters}{"27}
\newcommand\restr[2]{{% we make the whole thing an ordinary symbol
  \left.\kern-\nulldelimiterspace % automatically resize the bar with \right
  #1 % the function
  \vphantom{\big|} % pretend it's a little taller at normal size
  \right|_{#2} % this is the delimiter
  }}

%\graphicspath{{./graphics/}}%helpful if your graphic files are in another directory
\bibliographystyle{alpha} % the recommnded bibstyle

\usepackage[framemethod=tikz]{mdframed}
\usepackage{todonotes}

\usepackage{tikz}
\usepackage{appendix}
\usetikzlibrary{trees}
\usetikzlibrary{decorations.pathreplacing}
\usepackage{graphicx}
\usepackage[noabbrev]{cleveref}

\usepackage{paralist}
\usepackage[inline]{enumitem}
\newlist{myalist}{enumerate*}{1}
\setlist[myalist]{label=\textbf{(\arabic*)}}
\newlist{mylist}{enumerate*}{1}
\setlist[mylist]{label=\textbf{(\roman*)}}
\newlist{alphalist}{enumerate*}{1}
\setlist[alphalist]{label=\textbf{(\alph*)}}
%\usepackage{mathrsfs}
\usepackage{hyperref}
\usepackage{cleveref}
\newlist{inlinelist}{enumerate*}{1}
\setlist*[inlinelist,1]{%
	label=\textbf{(\roman*)},
}
%\usepackage{amssymb}
\DeclareRobustCommand{\subsetsim}{\protect{\mathrel{\substack{
				\textstyle\subset\\[-0.2ex]\textstyle\sim}}}}
\DeclareRobustCommand{\insim}{\mathrel{\substack{
			\textstyle\in\\[-0.2ex]\textstyle\sim}}}

%%\mathpzc
\DeclareFontFamily{OT1}{pzc}{}
\DeclareFontShape{OT1}{pzc}{m}{it}{<-> s * [1.10] pzcmi7t}{}
\DeclareMathAlphabet{\mathpzc}{OT1}{pzc}{m}{it}
\usepackage{amsthm}
\usepackage{thmtools,thm-restate}
\usepackage[makeroom]{cancel}
%\usepackage{mathtools}

\theoremstyle{plain}
%\newtheorem{bg}{bogus}
%\newtheorem{ex}{Example}

%\newtheorem{defn}{Definition}
%\newtheorem{lem}{Lemma}

\usepackage{mwe}
%\usepackage{subcaption}
\newsavebox{\verbbox}
\newsavebox{\verbboxa}
\newsavebox{\verbboxb}
\newsavebox{\verbboxc}


\usepackage{chngcntr}
\usepackage{apptools}
\usepackage{stmaryrd}
\makeatother


\makeatletter
\pgfdeclareshape{document}{
	\inheritsavedanchors[from=rectangle] % this is nearly a rectangle
	\inheritanchorborder[from=rectangle]
	\inheritanchor[from=rectangle]{center}
	\inheritanchor[from=rectangle]{north}
	\inheritanchor[from=rectangle]{south}
	\inheritanchor[from=rectangle]{west}
	\inheritanchor[from=rectangle]{east}
	% ... and possibly more
	\backgroundpath{% this is new
		% store lower right in xa/ya and upper right in xb/yb
		\southwest \pgf@xa=\pgf@x \pgf@ya=\pgf@y
		\northeast \pgf@xb=\pgf@x \pgf@yb=\pgf@y
		% compute corner of ‘‘flipped page’’
		\pgf@xc=\pgf@xb \advance\pgf@xc by-10pt % this should be a parameter
		\pgf@yc=\pgf@yb \advance\pgf@yc by-10pt
		% construct main path
		\pgfpathmoveto{\pgfpoint{\pgf@xa}{\pgf@ya}}
		\pgfpathlineto{\pgfpoint{\pgf@xa}{\pgf@yb}}
		\pgfpathlineto{\pgfpoint{\pgf@xc}{\pgf@yb}}
		\pgfpathlineto{\pgfpoint{\pgf@xb}{\pgf@yc}}
		\pgfpathlineto{\pgfpoint{\pgf@xb}{\pgf@ya}}
		\pgfpathclose
		% add little corner
		\pgfpathmoveto{\pgfpoint{\pgf@xc}{\pgf@yb}}
		\pgfpathlineto{\pgfpoint{\pgf@xc}{\pgf@yc}}
		\pgfpathlineto{\pgfpoint{\pgf@xb}{\pgf@yc}}
		\pgfpathlineto{\pgfpoint{\pgf@xc}{\pgf@yc}}
	}
}
\makeatother



%\usepackage{amssymb}
\DeclareMathAlphabet{\mathbbm}{U}{bbm}{m}{n}
\usepackage[stable]{footmisc}
\usepackage{braket}
\usepackage{comment}
\usepackage{varioref}
\usepackage{paralist}
\usepackage{booktabs}
\usepackage{listings}
\usepackage{xfrac}
\usepackage{comment}
\usepackage{listings}
%\lstset{language=Pascal}
\lstdefinelanguage{sq}
{morekeywords={SELECT, FROM, WHERE, AND, OR},
	sensitive=false,
	morecomment=[l]{//},
	morecomment=[s]{/*}{*/},
	morestring=[b]",
}

%\renewcommand{\doi}[1]{}
%\newcommand{\url}[1]{} %%Removing the DOI
%\renewcommand{\isbn}[1]{}
%\renewcommand{\issn}[1]{}

\newcommand{\phchapter}[1]{\chapter*{#1} \addcontentsline{toc}{chapter}{#1}}
\newcommand{\phsection}[1]{\section*{#1} \addcontentsline{toc}{section}{#1}}
\newcommand{\phsubsection}[1]{\subsection*{#1} \addcontentsline{toc}{subsection}{#1}}
\newcommand{\phsubsubsection}[1]{\subsubsection*{#1} \addcontentsline{toc}{subsubsection}{#1}}
\newcommand{\phparagraph}[1]{\paragraph*{#1} \addcontentsline{toc}{paragraph}{#1}}

\DeclareMathOperator{\avg}{avg}

\newenvironment{paralist}{\ifcomplete

	\else
	\begin{itemize}
		\fi}{
		\ifcomplete

		\else
	\end{itemize}
	\fi}
\newcommand{\paraitem}[2]{\ifcomplete
	\paragraph*{\textit{\textbf{#1}}}
	#2
	\else
	\item #1
	\fi
}
\newcommand{\summarized}[2]{\ifcomplete
	#2
	\else
	\item #1
	\fi
}
\newcommand{\conditionalcontent}[1]{\ifcomplete
	#1
	\else

	\fi
}
\newcommand{\highlinedcontent}[1]{\ifcomplete

	\else
	\quad [#1]
	\fi
}


\setlist[description]{leftmargin=*}

\newcommand{\UseVar}[1]{$\Braket{\textup{\textrm{#1}}}$}
\newcommand{\UseNVar}[2]{$\Braket{\textup{\textrm{#1}}}_{#2}$}
\newcommand{\DefVar}[1]{$\Braket{\textup{\textrm{#1}}} ::= $}
\newcommand{\NextDef}{$\qquad|\qquad$}

\usepackage{varwidth}
%\usepackage[dvipsnames]{xcolor}
\usepackage{lipsum}

%\usepackage{soul}

\usepackage[most]{tcolorbox}
\newenvironment{CenteredShadowboxListing}[1][]{%
	\tcbset{listing options={style=tcblatex,#1}}\tcbwritetemp}%
{\endtcbwritetemp%
	\tcbox[enhanced,arc=0pt,outer arc=0pt,top=1mm,bottom=1mm,left=1mm,right=1mm,
	boxrule=0.6pt,drop fuzzy shadow,before=\begin{center},after=\end{center}]%
	{\tcbusetemplisting}}

\makeatletter
\newcommand{\subalign}[1]{%
	\vcenter{%
		\Let@ \restore@math@cr \default@tag
		\baselineskip\fontdimen10 \scriptfont\tw@
		\advance\baselineskip\fontdimen12 \scriptfont\tw@
		\lineskip\thr@@\fontdimen8 \scriptfont\thr@@
		\lineskiplimit\lineskip
		\ialign{\hfil$\m@th\scriptstyle##$&$\m@th\scriptstyle{}##$\crcr
			#1\crcr
		}%
	}
}
\makeatother

%\newtheorem{definition}{Definition}

\graphicspath{{./}}

%\usepackage{amssymb}
\def\ojoin{\setbox0=\hbox{$\bowtie$}%
	\rule[0.2ex]{.27em}{.4pt}\llap{\rule[1ex]{.27em}{.4pt}}}
\def\leftouterjoin{\mathbin{\ojoin\mkern-2mu\Join}}
\def\rightouterjoin{\mathbin{\Join\mkern-3.4mu\ojoin}}
\def\fullouterjoin{\mathbin{\ojoin\mkern-5mu\Join\mkern-6mu\ojoin}}


	\usepackage{pifont}% http://ctan.org/pkg/pifont
	\newcommand{\cmark}{\ding{51}}%
	\newcommand{\xmark}{\ding{55}}%
\usepackage{threeparttable}

\usepackage{minitoc}
\setcounter{minitocdepth}{3}
\setcounter{secnumdepth}{3}
\dominitoc
\newcommand{\mstr}[1]{\textup{\color{webgreen}``#1''}}
\newcommand{\hstr}[1]{{\color{webgreen}\textup{``}#1\textup{''}}}
\newcommand{\defzt}[5]{\ell(#1)=[\texttt{#2}]\quad \xi(#1)=[\mstr{#3}]\quad \phi(#1,\mstr{#4})=[#5]}
\newcommand{\defzi}[3]{\ell(#1)=[\texttt{#2}]\quad \xi(#1)=[\mstr{#3}] }
\newcommand{\defSNU}[4]{\phi(#4,\ATTR)=[10#4,11#4]\\ \defzi{10#4}{Name}{#2}\quad \defzi{11#4}{Surname}{#3}}
\newcommand{\defSNE}[4]{\phi(#4,\SRC)=[#2]\quad \phi(#4,\DST)=[#3]}
\newcommand{\arialify}[1]{{\fontfamily{phv}\color{red}\textbf{#1}}}

\newcommand{\myTitle}{A new Nested Graph Model for Data Integration}
\newcommand{\myName}{Giacomo Bergami}
\newcommand{\myDepartment}{CSE Department}
\newcommand{\myUni}{Alma Mater Studiorum - Universit\`a di Bologna}
\title{\myTitle}
\author{\myName}
\coordinatore{Paolo Ciaccia}
\relatore{Danilo Montesi}
\titlepagebottomline{\myUni \qquad \qquad  \myDepartment}

\PassOptionsToPackage{pdftex,hyperfootnotes=false,pdfpagelabels}{hyperref}
\usepackage{hyperref}  % backref linktocpage pagebackref
\pdfcompresslevel=9
\pdfadjustspacing=1
\PassOptionsToPackage{pdftex}{graphicx}
\usepackage{graphicx}
\hypersetup{%
	%draft, % = no hyperlinking at all (useful in b/w printouts)
	colorlinks=true, linktocpage=true, pdfstartpage=3, pdfstartview=FitV,%
	% uncomment the following line if you want to have black links (e.g., for printing)
	%colorlinks=false, linktocpage=false, pdfstartpage=3, pdfstartview=FitV, pdfborder={0 0 0},%
	breaklinks=true, pdfpagemode=UseNone, pageanchor=true, pdfpagemode=UseOutlines,%
	plainpages=false, bookmarksnumbered, bookmarksopen=true, bookmarksopenlevel=1,%
	hypertexnames=true, pdfhighlight=/O,%nesting=true,%frenchlinks,%
	urlcolor=webbrown, linkcolor=RoyalBlue, citecolor=webgreen, %pagecolor=RoyalBlue,%
	%urlcolor=Black, linkcolor=Black, citecolor=Black, %pagecolor=Black,%
	pdftitle={\myTitle},%
	pdfauthor={\textcopyright\ \myName, \myUni, \myDepartment},%
	pdfsubject={},%
	pdfkeywords={},%
	pdfcreator={pdfLaTeX},%
	pdfproducer={\myTitle},%
}

\usepackage{subcaption}

%%
%%\titleformat{\part}[display]
%%{\normalfont\centering\huge}%
%%{\thispagestyle{empty}\partname~\textsc{\thepart}}{1em}%
%%{\color{Maroon}\spacedallcaps}[\bigskip\normalfont\normalsize\color{Black}\begin{quote}\ct@parttext\end{quote}]

%\newenvironment{aenum}{\begin{enumerate*}[label=\textit{({{\color{red}\alph*})}]}{\end{enumerate*}}

\usepackage{inconsolata}

%% ALGORITHMIC PACKAGE
\usepackage{algorithm}
\usepackage[noend]{algpseudocode}
\makeatletter
\renewcommand{\ALG@beginalgorithmic}{\small}
\makeatother
\makeatletter
\newcounter{ALC@tempcntr}% Temporary counter for storage
\newcommand{\LCOMMENT}[1]{%
	\setcounter{ALC@tempcntr}{\arabic{ALC@rem}}% Store old counter
	\setcounter{ALC@rem}{1}% To avoid printing line number
	\item \{#1\}% Display comment + does not increment list item counter
	\setcounter{ALC@rem}{\arabic{ALC@tempcntr}}% Restore old counter
}%


\usepackage{listings}
\definecolor{eclipseBlue}{RGB}{42,0.0,255}
\definecolor{eclipseGreen}{RGB}{63,127,95}
\definecolor{eclipsePurple}{RGB}{127,0,85}
\lstset{basicstyle=\ttfamily,%
	%backgroundcolor=\color[rgb]{0.85,0.85,0.86},%
	frame=single,
	framerule=0pt,
	xleftmargin=\fboxsep,
	xrightmargin=\fboxsep,
  commentstyle=\color{eclipseGreen}, % style of comments
  keywordstyle=\color{eclipsePurple}, % style of keywords
  stringstyle=\color{eclipseBlue},
  breaklines=true,
  postbreak=\raisebox{0ex}[0ex][0ex]{\ensuremath{\color{red}\hookrightarrow\space}}
}
\captionsetup[lstlisting]{position=top}


\newcommand{\jsem}[1]{[\! [#1]\!]}
\newcommand{\sem}[2][M\!,g]{\mbox{ $[\![ #2 ]\!]^{#1}$}}

%%% FONTS
\usepackage{MnSymbol}
\usepackage{pdftexcmds}
%\DeclareFontFamily{U}{cbgreek}{}
%\DeclareFontShape{U}{cbgreek}{m}{n}{
%	<-6>    grmn0500
%	<6-7>   grmn0600
%	<7-8>   grmn0700
%	<8-9>   grmn0800
%	<9-10>  grmn0900
%	<10-12> grmn1000
%	<12-17> grmn1200
%	<17->   grmn1728
%}{}
%\DeclareFontShape{U}{cbgreek}{bx}{n}{
%	<-6>    grxn0500
%	<6-7>   grxn0600
%	<7-8>   grxn0700
%	<8-9>   grxn0800
%	<9-10>  grxn0900
%	<10-12> grxn1000
%	<12-17> grxn1200
%	<17->   grxn1728
%}{}
%
%\DeclareRobustCommand{\qoppa}{%
%	\text{\usefont{U}{cbgreek}{\normalorbold}{n}\symbol{19}}%
%}
%\DeclareRobustCommand{\Qoppa}{%
%	\text{\usefont{U}{cbgreek}{\normalorbold}{n}\symbol{21}}%
%}
%\makeatletter
%\newcommand{\normalorbold}{%
%	\ifnum\pdf@strcmp{\math@version}{bold}=\z@ bx\else m\fi
%}
%\makeatother
%
\usepackage{cases}

\lstdefinelanguage{Script}{
	keywords={if, then, else, substring, remove, from, in, map, select, foldl, phi, xi, ell, id},
	literate={+}{{{\color{RoyalBlue}+}}}1
	{-}{{{\color{RoyalBlue}-}}}1
	{/}{{{\color{RoyalBlue}/}}}1
	{*}{{{\color{RoyalBlue}*}}}1
	{@}{{{\color{RoyalBlue}@}}}1
	{&&}{{{\color{RoyalBlue}\&\&}}}1
	{||}{{{\color{RoyalBlue}||}}}1
	{not}{{{\color{RoyalBlue}not}}}1
	{>}{{{\color{RoyalBlue}>}}}1
	{>=}{{{\color{RoyalBlue}>=}}}1
	{<}{{{\color{RoyalBlue}<}}}1
	{<=}{{{\color{RoyalBlue}<=}}}1
	{==}{{{\color{RoyalBlue}==}}}1
	{!=}{{{\color{RoyalBlue}!=}}}1
	{:=}{{{\color{RoyalBlue}:=}}}1
	{:}{{{\color{RoyalBlue}:}}}1
	{tt}{{{\color{RoyalBlue}tt}}}1
	{ff}{{{\color{RoyalBlue}ff}}}1,
	%ndkeywords={+, -, /, *, ++, @, &&, ||, not, ==, !=, <=, >=, >, <, :=, .},
	sensitive=true,
	morestring=[b]",
	frameshape={RYR}{Y}{Y}{RYR}
}



\lstdefinelanguage{JavaScript}{
  keywords={typeof, new, true, false, catch, function, return, null, catch, switch, var, if, in, while, do, else, case, break},
  ndkeywords={class, export, boolean, throw, implements, import, this},
  sensitive=false,
  comment=[l]{//},
  morecomment=[s]{/*}{*/},
  morestring=[b]',
  morestring=[b]"
}

\lstdefinelanguage{sparql}{
morecomment=[l][]{\#},
morestring=[b][]\",
morekeywords={BIND,URI,CONCAT,SELECT,CONSTRUCT,DESCRIBE,ASK,WHERE,FROM,NAMED,PREFIX,BASE,OPTIONAL,FILTER,GRAPH,LIMIT,OFFSET,SERVICE,UNION,EXISTS,NOT,BINDINGS,MINUS,a},
sensitive=true
}

\lstdefinelanguage{cypher}{
	morekeywords={MATCH,RETURN,WHERE,DISTINCT,WITH,CREATE,COUNT,AS,UNION,ALL,is,null,NOT,AND,OR},
	sensitive=true,
	morecomment=[l]{//}, % l is for line comment
}
\lstdefinelanguage{gremlin}{
	morekeywords={g,V,match,as,hasLabel,out,has,inE,outV,as,between,select,groupCount,TinkerGraph,io,IoCore,readGraph,traversal,values,in,count,is},
	sensitive=true
}
\lstdefinelanguage{biql}{
	morekeywords={CREATE,SELECT,FROM,WHERE,AS,count},
	sensitive=true
0}
\lstdefinelanguage{nosql}{
	morekeywords={create,class,extends,edge,vertex,select,from,where,in,or,and,not,out,delete,insert,into,as},
	sensitive=false
}

\usepackage{ifsym}
%\usepackage{amssymb}

%\DeclareRobustCommand{\ojoin}{\rule[0.10ex]{.3em}{.4pt}\llap{\rule[1.40ex]{.3em}{.4pt}}}
%\newcommand{\joino}{\bowtie}
%\newcommand{\leftouterjoin}{\tiny \textifsym{d|><|}}
%\newcommand{\rightouterjoin}{\tiny \textifsym{|><|d}}
%\newcommand{\fullouterjoin}{\tiny \textifsym{d|><|d}}

%% Redefining XML Colors
\definecolor{cyan}{rgb}{0.0,0.6,0.6}
\lstdefinelanguage{XML}
{
  morestring=[b]",
  morestring=[s]{>}{<},
  morecomment=[s]{<?}{?>},
  stringstyle=\color{eclipseGreen},
  identifierstyle=\color{eclipseBlue},
  keywordstyle=\color{cyan},showstringspaces=false,
  morekeywords={xmlns,xsi,noNamespaceSchemaLocation,type,id,x,y,source,target,version,tool,transRef,roleRef,objective,eventually}% list your attributes here
}
\lstdefinelanguage{antlr}{
	moredelim=[s]{'}{'},% single quotes in green
	moredelim=*[s]{options}{\}},%  options in black (until trailing })
	comment=[l]{\#},%                                  define // comment
	morekeywords={%
		script,expr,BOOL,VARNAME,FUNVAR,EscapedString,NUMBER,WS%                                            literal strings listed here
	},
	alsoletter={:,|,;},%
	morekeywords={:,|,;}%                                 define the special characters
}
\lstdefinelanguage{AQL}{
	morekeywords={
	FOR,IN,COLLECT,INTO,RETURN
	},
	morestring=[b]",
	stringstyle=\color{eclipseGreen},
	morecomment=[l]{--}, % l is for line comment
}

\renewcommand{\epigraphsize}{\small}
\setlength{\epigraphwidth}{0.6\textwidth}
\renewcommand{\textflush}{flushright}
\renewcommand{\sourceflush}{flushright}
% A useful addition
\newcommand{\epitextfont}{\itshape}
\newcommand{\episourcefont}{\scshape}

\makeatletter
\newsavebox{\epi@textbox}
\newsavebox{\epi@sourcebox}
\newlength\epi@finalwidth
\renewcommand{\epigraph}[2]{%
  \vspace{\beforeepigraphskip}
  {\epigraphsize\begin{\epigraphflush}
   \epi@finalwidth=\z@
   \sbox\epi@textbox{%
     \varwidth{\epigraphwidth}
     \begin{\textflush}\epitextfont#1\end{\textflush}
     \endvarwidth
   }%
   \epi@finalwidth=\wd\epi@textbox
   \sbox\epi@sourcebox{%
     \varwidth{\epigraphwidth}
     \begin{\sourceflush}\episourcefont#2\end{\sourceflush}%
     \endvarwidth
   }%
   \ifdim\wd\epi@sourcebox>\epi@finalwidth
     \epi@finalwidth=\wd\epi@sourcebox
   \fi
   \leavevmode\vbox{
     \hb@xt@\epi@finalwidth{\hfil\box\epi@textbox}
     \vskip1.75ex
     \hrule height \epigraphrule
     \vskip.75ex
     \hb@xt@\epi@finalwidth{\hfil\box\epi@sourcebox}
   }%
   \end{\epigraphflush}
   \vspace{\afterepigraphskip}}}
\makeatother


\DeclareRobustCommand{\hlcyan}[1]{{\sethlcolor{cyan}\hl{#1}}}

%% Nested graph symbols
\usepackage[safe]{tipa}
\newcommand{\bin}{\texttt{bin}}
\newcommand{\smatch}{\textup{\textturnw}}
\newcommand{\nested}{\textup{\textrtailn}}
\newcommand{\ngraph}{\textup{\textcrg}}
\newcommand{\genp}{\textup{\textbabygamma}}
\newcommand{\ONTA}{\mstr{Entity}}
\newcommand{\RELA}{\mstr{Relationship}}
\newcommand{\ATTR}{\mstr{Attribute}}
\newcommand{\SRC}{\mstr{src}}
\newcommand{\DST}{\mstr{dst}}
\DeclareDocumentCommand{\fullnested}{ O{\nested} O{\ngraph} }{\ifthenelse{\equal{#1}{}}{\nested}{#1} = (\ifthenelse{\equal{#2}{}}{\ngraph}{#2},O,\ell,\xi,\phi)}

\declaretheorem[name=Lemma,numberwithin=section]{rlem}

\usepackage{letltxmacro}
\newcommand{\scriptline}[1]{\text{\lstinline[{language=Script}]{#1}}}
\usepackage{marginnote}

%\DeclareRobustCommand*{\scriptline}{%
%	\ifmmode
%	\let\SavedBGroup\bgroup
%	\def\bgroup{%
%		\let\bgroup\SavedBGroup
%		\hbox\bgroup
%	}%
%	\fi
%	\SavedLstInline
%}

\AtEndEnvironment{definition}{\qed}
%\usepackage{scrextend}
\usepackage{amssymb}
\usepackage{extarrows}

\newfloat{lstfloat}{htbp}{lop}
\floatname{lstfloat}{Listing}
\usepackage{cleveref}

\let\Sectionmark\sectionmark
\def\sectionmark#1{\def\Sectionname{#1}\Sectionmark{#1}}

\def\igobble#1 {}

%%%%%%%%%%%%%%%%%%%%%%
%%%%%% GLOSSARY %%%%%%
%%%%%%%%%%%%%%%%%%%%%%
\usepackage[savewrites,seeautonumberlist]{glossaries}
\newglossarystyle{mystyle}{%
	\glossarystyle{long}%
	\renewenvironment{theglossary}%
	{\begin{longtable}{p{3cm}p{\glsdescwidth}}}%
		{\end{longtable}}%
}
\makeglossaries

\newglossaryentry{transcodingOperator}{%
	name=\ensuremath{\stigma},
	description={Value Transcoding Function}
}
\newcommand{\transcoding}{\stigma\index{\stigma}}

\newglossaryentry{data}{%
	name=\ensuremath{D},
	description={Data}
}
\newcommand{\data}{D\index{D}}

\newglossaryentry{model}{%
	name=\ensuremath{M},
	description={Model}
}
\newcommand{\model}{M\index{M}}

\newglossaryentry{metamodel}{%
	name=\ensuremath{MM},
	description={Model}
}
\newcommand{\metamodel}{{MM}\index{MM}}

\newglossaryentry{abstr}{%
	name=\ensuremath{\alpha},
	description={Instance Of, Abstraction}
}
\newcommand{\abstr}{\alpha\index{\alpha}}
\newcommand{\abstrTHIS}{\alpha\index{\alpha|textbf}}

\newglossaryentry{translate}{%
	name=\ensuremath{\tau},
	description={(Data representation) Translation}
}
\newcommand{\ttransl}{\tau\index{\tau}}
\newcommand{\ttranslTHIS}{\tau\index{\tau|textbf}}
\newcommand{\ttranslIndex}[1]{\tau\index{\tau!#1}}

\newglossaryentry{QTranslate}{%
	name=\ensuremath{\Qoppa},
	description={(Query) Translation}
}
\newcommand{\qtransl}{\Qoppa\index{\Qoppa}}
\newcommand{\qtranslTHIS}{\Qoppa\index{\Qoppa|textbf}}
\newcommand{\qtranslIndex}[1]{\Qoppa\index{\Qoppa!#1}}

\newglossaryentry{langg}{%
	name=\ensuremath{\mathcal{L}},
	description={Query Language over a MetaModel ($\mathcal{L}_{MM}$)}
}
\newcommand{\langg}{\mathcal{L}\index{\mathcal{L}}}
\newcommand{\lang}{\langg}

\newglossaryentry{alignment}{%
	name=\ensuremath{A},
	description={(Ontological) Alignment}
}
\newcommand{\alignment}{A\index{A}}


\newcommand{\GSMindex}[1]{\index{General Semistructured Model!#1}}





\makeatletter
\newenvironment{tsubarray}[1]{%
	\vcenter\bgroup
	\Let@ \restore@math@cr \default@tag
	\baselineskip\fontdimen10 \scriptfont\tw@
	\advance\baselineskip\fontdimen12 \scriptfont\tw@
	\lineskip\thr@@\fontdimen8 \scriptfont\thr@@
	\lineskiplimit\lineskip
	\check@mathfonts
	\ialign\bgroup\ifx c#1\hfil\fi
	\normalfont\fontsize\sf@size\z@\selectfont\ignorespaces##\unskip\hfil\crcr
}{%
	\crcr\egroup\egroup
}
\makeatother
\newcommand{\tsub}[1]{\begin{tsubarray}{l}#1\end{tsubarray}}

\usepackage{amsmath}