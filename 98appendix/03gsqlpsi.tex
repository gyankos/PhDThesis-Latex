\chapter{Expressing containment functions in \texttt{script}}\label{app:scriptnote}

This appendix provides the script notation for the mathematical notation used in the GSQL definitions. We chose to use the latter instead of \texttt{script} due to the fact that this notation is more compact and more readable. On the other hand, a \texttt{script} implementation of such functions shows how such definitions can be implemented in any system. Therefore, this appendix is going to provide the implementation of the aforementioned mathematical notation.

{\par\LARGE Script for $\psi_\cup$\par}

The associated \texttt{script} expression can be defined as follows: 
\begin{lstlisting}[language=script]
fold [o.phi,
      k-> { {{k[0][0][1], k[1][k[0][0][1]]++k[0][1]}} ++ 
            select(k[1] : y -> { not (y[0] == k[0][1])})
          },
      {}
     ] 
\end{lstlisting}
Please remember that \texttt{k} is a pair, where \texttt{k[0]} is the current element of \scriptline{o.phi} while \texttt{k[0][1]} is the accumulated value that, in our case, is the step-by-step reformulation of the containment.

{\par\LARGE Script for $\psi_\backslash$\par}

Before providing the definition of $\psi_\backslash$ in \texttt{script}, we must define some script utility functions, such as some set (or, as in this case, list) operations. 

\begin{lstlisting}[language=script]
difference = x -> select (x[0] : y -> { not (y in x[1])})
intersect = x -> select (x[0] : y -> { y in x[1]})
distinct = x -> {
	fold {x,
	      y->{if (y[0] in y[1]) then y[1] else [y[0]]++y[1]},
	      {}
	     }
}
\end{lstlisting}

We can also define some further shorthands for accessing each object's attributes. Let us remember that for each element \texttt{x} in \scriptline{o.phi}, \texttt{x[0]} represents the attribute $a$ associated to the containment, and hence \texttt{x[0][0]} represent the original operand, while \texttt{x[0][1]} represent the attribute appearing in the original operand. 

\begin{lstlisting}[language=script,mathescape=true]
keys $\eqdef$ (distinct (map (o.phi : x -> x[0][1])))
operands $\eqdef$ (distinct (map (o.phi : x -> x[0][0])))
\end{lstlisting}

Moreover, the following function returns a list of pairs \texttt{\{p, l\}} for every containment $\phi(g,[\texttt{y},\texttt{p}])=\texttt{l}$ found in the disjunct united nested graph belonging to the \texttt{y}-th operand.

\begin{lstlisting}[language=script]
getOperandList = y -> { map(select(o.phi : x -> {x[0][0] = y}) : 
                            x -> {x[0][1], x[1]}) 
                      }
\end{lstlisting}

By combining some of the previous functions, we obtain the final desired result:

\begin{lstlisting}[language=script,]
map(keys : x -> {{x, 
                  (difference {(getOperandList 0)[x], 
                               (getOperandList 1)[x]
                              })
                 }
                }
   )
\end{lstlisting}

{\par\LARGE Script for $\psi_\cap$\par}

Similarly to the previous step, we have to perform the intersection over all the operands over the common set of attributes.

\begin{lstlisting}[language=script]
map(keys : x -> {
             (fold { remove operands[0] in operands, 
                     y -> { (intersect {(getOperandList (y[0]))[x], 
                                        y[1]
                                       })
                          },
                     (getOperandList (operands[0]))[x]   
                   })
})
\end{lstlisting}

%\begin{lstlisting}[language=script]
%(fold {attributes,
%cp -> { (fold {remove operands[0] in operands,
%cpp -> {(intersect {(getOperandList cpp[0])[cp[0]], cpp[1]}) [cp[0]]},
%cp[1]  
%}
%) },
%getOperandList (operands[0]) })
%\end{lstlisting}
