\phparagraph{Extended Property Graph Model (EPGM)}\label{php:EPGM}
\index{graph!extended property graph}
The need of representing both graphs and graph collections for handling pattern matching queries presented for GraphQL \cite{He2007}\index{GraphQL!He} brought to the definition of data models where both representations are provided. Therefore, the property graph data model requires to be extended because property graphs do not natively support graph collections. For this reason both GRADE \cite{GRAD,Ghrab2015} and EPGM \cite{apacheflink,JunghannsPR17} were introduced: within this thesis we're going to describe only the latter one due to its practical implementation in \textsc{Gradoop}. This data model can be re-defined\footnote{From now on, the data models are re-defined using the same notation and terminology, in order to remark the similarities with the previous models. In particular, $\phi$ and $\omega$ are introduced as containment functions for vertices and edges, and $\lambda$ is used to associate to each edge its source and destination vertices. Later on, we're only going to use $\phi$ for both vertex and edge containments.} as follows:

\begin{definition}[EPGM Database]
	\index{graph!EPGM|textbf}
	An EPGM database $DB = ( V, E, L, K, T, A,\lambda,\phi,\omega, \kappa )$ consists of vertex set $V\subseteq \mathbb{N}$, edge set $E\subseteq \mathbb{N}$ and a set of logical graphs $L\subseteq \mathbb{N}$ such that those identifiers' sets are  pairwise disjoint. An edge $e\in E$ is mapped to its source and target vertices through the $\lambda$ function, e.g. $\lambda(e) = ( s,t )$, where $\lambda\colon E\to V^2$.  To each \textbf{logical graph}\index{logical graph!within EPGM} $g \in L$ is  associated a set of vertices and edges through the $\phi$ and $\omega$ functions, such that $\phi(g)\subseteq V$ and $\omega(g)\subseteq E$, and each extracted edge connects edges within $\phi(g)$ ($\forall e\in \omega(g).\lambda(e)\in\phi(g)^2$). Vertex, edge and graph properties are defined by key set $K$, value set $A$ and mapping $\kappa : \mathbb{N} \times K \rightarrow A$. Labels $T$ are expressed as a value $A$ associated to a key $\tau\in K$.
\end{definition}

Even though this data model was  used to express graph aggregations,  neither logical graphs nor vertices could be directly used to describe the structural content of summarized graphs.  Logical graphs were not used for linking aggregated informations because EPGM does not allow any primitive object acting as a relation between either logical graphs or vertices. Moreover, neither vertices nor edges can be used for structural aggregation because, as for the property graph model, $A$ values can represent neither object identifiers nor collections. As a consequence, ancillary ``super-vertices'' and ``super-edges'' were used to store the result of the aggregation, consisting respectively of a collection of vertices and edges. Therefore, even this definition fails at representing nested graphs, where it is required that each component contains a whole graph as represented in the former Figure \vref{fig:firstnested}.