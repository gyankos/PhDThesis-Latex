\section{Classifying Graph Query Languages}\label{sec:dbqlang}

%\chapter{Languages, Systems and Libraries}\label{sec:dbqlang}
%\epigraph{``\textup{Chi insegna convien tenti mille strade, dia mille stimoli, usi mille termini, pensi mille modi
%	ed adoperi ogni maniera, che anche i sassi, per così dire, ne ricevano dell'impressione.}''
%
%	``\textit{Those who teach must attempt a thousand roads, give thousands of stimuli, use a thousand words,
%		think in all possible ways and work every way, so that even stones, so to speak, could be
%		impressed}''.}{--- Francesco Antonio Marcucci\\ (1717 – 1798)}

Graph Query Languages can be categorised in three main classes;
\begin{alphalist}
	\item the first class  tries to find a
	possible match for a specific traversal  or for extracting all the subgraphs that match a given pattern (Section \vref{sec:gtl}): as a consequence such graph
	queries do not necessarily manipulate the graph data structure (Except for GraphLOG).
	\item The second class are graph grammars, that are used to rewrite parts of a same graph by using rewriting rules (Section \vref{subsec:ggram}).
	\item The third class are graph algebras, that extend the previous operations with set and simil-relational algebra operations for graphs (Section \vref{cite:galg}).
	\item The last class of languages are ``proper'' graph query languages, that include within their expressive power all the aforementioned class of operations (Section \vref{subsec:proper}).
\end{alphalist}

