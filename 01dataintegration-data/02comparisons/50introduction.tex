\section{Classifying Graph Query Languages}\label{sec:dbqlang}

%\chapter{Languages, Systems and Libraries}\label{sec:dbqlang}
%\epigraph{``\textup{Chi insegna convien tenti mille strade, dia mille stimoli, usi mille termini, pensi mille modi
%	ed adoperi ogni maniera, che anche i sassi, per così dire, ne ricevano dell'impressione.}''
%
%	``\textit{Those who teach must attempt a thousand roads, give thousands of stimuli, use a thousand words,
%		think in all possible ways and work every way, so that even stones, so to speak, could be
%		impressed}''.}{--- Francesco Antonio Marcucci\\ (1717 – 1798)}

Contrariwise to current graph query languages' surveys \cite{AnglesABHRV17}, we're going to classify them not only from their expressiveness and ability to perform several four of traversal and matching queries but also by their ability to generate new graph data. Therefore, graph Query Languages can be categorised in three main classes:
\begin{enumerate}
	\item The first class  tries to find a
	possible match for a specific traversal  or for extracting all the subgraphs that match a given pattern (Section \vref{sec:gtl}): as a consequence such graph
	queries do not necessarily manipulate the graph data structure (Except for GraphLOG). This implies that such query languages must traverse the data structure, and it should express data properties by accessing the pieces of informations stored in graphs.
	\item The second class are graph grammars, that are used to rewrite parts of a same graph by using rewriting rules (Section \vref{subsec:ggram}). These languages add the capability of adding and removing new vertices and edges, which do not necessarily depend on previous data.
	\item The third class are graph algebras, that extend the previous operations with set and simil-relational algebra operations for graphs (Section \vref{cite:galg}). These languages permit n-ary operators and the modification (projection, extension) of already-existing objects.
	\item The last class of languages are ``proper'' graph query languages, that include within their expressive power all the aforementioned class of operations (Section \vref{subsec:proper}).
\end{enumerate}

