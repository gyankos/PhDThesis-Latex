\phparagraph{Graph Data models for roll-up and drill-down operations}
As outlined in the Introduction at page \pageref{molap}, graph representation have recently became a favourite representation for multidimensional data. 
Current literature uses two different approaches for extending
% Two different approaches have been used to extend 
graph databases to support nesting operations:
some try to overcome graph data structure limitations by extending their query languages, while others try to extend the data structures used for both input and intermediate computations. Among the first type of approaches, \cite{Etcheverry2012} proposes the definition of a RDF  vocabulary over which the OLAP  cube can be defined. On top of this ``structured'' RDF graph, an algorithm generates the SPARQL query that will allow to perform either the roll-up or the drill-down operation. This implies that each possible computation over the data view has to be always recomputed on top of the raw data as in ROLAP systems, thus thwarting the benefits of updating the intermediate query result. On the other hand, the last type of approaches has been recently widely investigated  and seems to be more promising with regard to optimization techniques. In these approaches \cite{Tian20085,ChenYZHY08,Qu2011}, graph data structures are associated with external graph indices and, thus, allow to connect one graph to a broader one with respect to the roll-up query. As a consequence, these solutions do not allow to freely expand any aggregate components at the same time but can only backtrack the aggregation to a previous known state.


%%As outlined in the Introduction at page \pageref{molap}, graph representation have recently became a favourite representation for multidimensional data. Two different approaches have been used to extend current graph data model into nested representations:
%%the first ones try to overcome to the basic graph data structure limitations by simply extending the query language, while the other ones try to extend the data structures that are used for both input and intermediate computations. 


%%Among the first type of approaches, the one outlined in \cite{Etcheverry2012} propose to define a RDF\index{RDF} vocabulary over which the OLAP\index{OLAP} cube can be defined in RDF\index{graph!RDF}. On top of this ``structured'' RDF graph, an algorithm generates the SPARQL query that will allow to perform either the roll-up and the drill-down operation\footnote{In particular, let's remember that \textit{roll-up} operations aggregate multidimensional data along one or more dimensions while computing aggregation functions over the aggregated values. On the other hand, the \textit{drill-down} operation is the inverse operation of the first one, allowing the user to navigate among different abstraction levels, ranging from the most summarized to the most detailed one.}. This later approach implies that each possible computation has to be always recomputed on top of the row data like for classical ROLAP systems: as a consequence, this MOLAP approach does not benefit from the specific RDF representation, that is not able to represent different aggregation levels and to store intermediate computations.

%%The last type of approaches have been recently widely investigated, and seems to be more promising with respect to optimization techniques: in \cite{Tian20085,ChenYZHY08,Qu2011} graph data structures are associated with external graph indices, thus allowing to connect one graph  to its broader one with respect to the roll-up query. As a consequence, these solutions do not allow to freely expand any aggregated component at a time, but they can only backtrack the aggregation to a previous known state. %Some further details are going to be provided on Chapter \vref{cha:nesting}, where such operator will be implemented on a specific algorithm.
As it will be showed in Chapter \ref{cha:graphsdef}, in order to meet such goals the nesting indices are going to be directly embedded within the definition of the nested (graph) data model, thus allowing to extend all the aforementioned approaches.