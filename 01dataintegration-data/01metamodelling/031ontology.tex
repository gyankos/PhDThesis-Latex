%%%%%
\section{Multi-database integration}\label{sec:integsurvey}
In this section we're interested to attack the \textbf{multi-database integration} problem, that can be stated as follows: \textit{given a global schema $G$ of my integrated system and a query\index{query|textbf} $q$ written in a  \textsc{Language} $\langg_G$, evaluate such query among multiple data-sources $\mathcal{D}=\Set{D_1,\dots, D_n}$ having   different schemas and data representations and provide the final results in a reconciled representation}. This definition is general enough to include both distributed and federated databases data integration.

In order to proficiently solve this problem, we're going to see that ontologies (Section \ref{sec:ontology}), generalizing data schemas  \cite{GangemiP13}, attack the schema matching problem (Section \ref{subsec:ontaling}) on different data sources with several schemas and representations. Last, we address the problem of evaluating query $q$ on either the original data sources or on the global schema (Section \ref{subsec:queryrw}).

\subsection{Preliminaries: Description Logic and Ontologies}\label{sec:ontology}
When compared with other data models such as the three world relational model for data mining \cite{Calders2006}, MOF\index{MetaObject Facility} also shows its inability to outline and suggest which are the relevant operations to either manipulate the data or to perform assertions on it. Ontologies are now introduced in particular for the latter reason, thus allowing to assert properties over \textit{objects} and \textit{relations} at the \textsc{Data} level, as required by the \textsc{Model} level. Literature provides the following generic definition of an ontology:

\begin{definition}
	\label{def:taonta}
	An \textbf{ontology}\index{ontology|textbf} \cite{Allemang2011} is a semantic interpretation of a model establishing relations among model's types and defining generic inference rules. Moreover ``\textit{an ontology plays a role of a semantic domain in denotational semantics}'' \cite{saeki} and refers to concepts or entities, that are language  and representation independent \cite{mathmeta}.
\end{definition}

Even though the MOF characterization clearly puts the ontology at the \textsc{Language} level, the unclear literature characterization has the problem of distinguishing \textsc{Model}s$\index{M}$ from ontologies \cite{terrasse} because their distinction is ``\textit{diverse and frequently contradictory}'' \cite{mathmeta}. On the other hand, other researchers cited in \cite{saeki} claim that the aim of distinguishing \textsc{Model}s from ontologies is to separate the representation of the information from the description of the data collected inside it. On the other hand, \cite{saeki} distinguishes the \textsc{Model} from the ontology level by defining a semantic mapping function associating each element of the \textsc{Model} to a collection of the elements of the ontology. Within the MOF characterization, such definition matches with the $\alpha$ function, where each element of the ontology is provided within one single formula predicating the properties associated to the type.

Among all the possible classes of  \textsc{Language} expressing ontologies (such as \textbf{frame-based} or \textbf{first order logic-based} languages \cite{Corcho00}), we chose to  describe ontologies using Description Logic (DL)\index{description logic} languages. This choice is due to the wide diffusion of OWL, a member of the DL language family, for Semantic Web applications. DL is also widely dealt in current literature, such that their notation is even adopted for expressing generic concepts in data integration literature \cite{euzenat2013d}. As a consequence, we provide a DL-biased definition for an ontology:

\begin{definition}[Description Logic Ontology]\label{def:dlonta}
	Given a set $I=\Set{i_1,\dots,i_n,\dots}$ of individuals, a \textbf{description logic ontology} $O$ \index{ontology!description logic} \index{description logic|see {ontology}} is a pair $(\mathcal{A},\mathcal{T})$, where both properties of individuals and binary relations between such individuals are expressed.

In particular, $\mathcal{A}$ is called \textbf{ABox} because it contains the following ``sorts'' of term, called \textbf{axioms}:
	\[t_{\mathcal{A}}:=C(i)\;|\;R(i,i')\;|\;C_1\sqsubseteq C_2\;|\;C_1\equiv C_2\;|\;R_1\sqsubseteq R_2\;|\; R_1\equiv R_2\]
	where $C$ describe \textbf{concepts} (\textit{types}) characterizing the individuals, and $R$ represent \textbf{roles} (\textit{relationships}) among source ($i$) and destination ($i'$) individuals. Moreover, inclusions ($\sqsubseteq$) and equivalence ($\equiv$) between the concepts could be also defined among concepts or roles.

$\mathcal{T}$ represents the \textbf{TBox} containing all the assertions and properties concerning the statements within the ABox. Such assertions are expressed within description logic languages, which expressive power and computational complexity may vary depending on the constructors of choice. An example of such constructors for the SROIQ language is provided in Table \ref{tab:SROIQ}.
\end{definition}

As a consequence, the \textsc{Model} is a part of the ontology ($M_O$) describing the collection of all the types that could be represented. On the other hand, ABox and TBox statements do not allow to transform individuals into others via transcoding functions $\transcoding$, because such logic focuses more on expressing properties over existing data than on showing which operations shall be used on top of such data.

\begin{table}
\centering
	\begin{tabular}{l|l|ll}
	\toprule
	& Description & Syntax ($\cdot$) & Semantics ($\cdot^{\mathcal{I}}$)\\
	\midrule
%\textbf{Individuals} 	& & \\
%individual name &	$o$ &	$o^{\mathcal{I}} = o$ \textbf{s.t.} $o\in D_O$\\
\parbox[t]{2mm}{\multirow{3}{*}{\rotatebox[origin=c]{90}{\textbf{Roles}}}} & atomic role &  $R$ & $\{r\in D_R|\alpha(r)=R\}$ \\
& inverse role & $R^{-}$ & $\{(x,y)|(y,x)\in R^{\mathcal{I}}\}$\\
& universal role & $U$ & 	$D_O\times D_O$\\
\midrule
\parbox[t]{2mm}{\multirow{11}{*}{\rotatebox[origin=c]{90}{\textbf{Concepts}}}} & atomic concept & $A$ & $\{o\in D_O| \alpha(o)=A\}$\\
& intersection &	$C\sqcap D$ & $C^{\mathcal{I}}\cap D^{\mathcal{I}}$\\
& union &	$C\sqcup D$ & $C^{\mathcal{I}}\cup D^{\mathcal{I}}$\\
& complement &	$\neg C$ & $D_O\backslash C^{\mathcal{I}}$\\
& top & $\top$ & $D_O$\\
& bottom & $\bot$ & $\emptyset$ \\
& existential restriction 	& $\exists R. C$ & $\{o\in D_O|  \exists o'\in D_O. (o,o')\in R^{\mathcal{I}} \wedge o'\in C^{\mathcal{I}}\}$ 	\\
& universal restriction 	& $\forall R. C$ & $\{o\in D_O|  \forall o'\in D_O. (o,o')\in R^{\mathcal{I}} \Rightarrow o'\in C^{\mathcal{I}}\}$ 	\\
& at-least restriction & $\geq n\; R.C$ & $\Set{o\in D_O|\left|\Set{o'\in D_O|(o,o')\in R^{\mathcal{I}}\wedge o'\in C^{\mathcal{I}}}\right|\geq n }$ \\
& at-most restriction & $\leq n\; R.C$ & $\Set{o\in D_O|\left|\Set{o'\in D_O|(o,o')\in R^{\mathcal{I}}\wedge o'\in C^{\mathcal{I}}}\right|\leq n }$ \\
& reflexivity & $\exists R. \textsc{Self}$ & $\Set{x\in D_O| (x,x)\in R^{\mathcal{I}}}$\\
& nominal & $\{o\}$ & $\{o|o\in D_O\}$\\
\bottomrule
	\end{tabular}
	\caption{Syntax and semantics of SROIQ constructors using \textsc{Data} as a domain for the interpretation under the CWA. SROIQ is the DL language expressing the OWL2.}
	\label{tab:SROIQ}
	\end{table}


Another difference between Description Logic and standard query  \textsc{Language} is that the preferred TBox interpretation  relies on the ``open world assumption'' (OWA)\index{OWA} \cite{Baader2010}. This implies that the truth value of the DL statements may be true irrespectively of whether or not it is known to be true within the ABox, thus opposing their query interpretation to the closed world assumption (CWA)\index{CWA}, where only the represented data are assumed to be true, thus describing a ``negation as failure'' approach (NAF, \cite{REN2010692}). As a consequence,  OWA semantics for {very expressive languages leads to an undecidable evaluation of the TBox assertions, thus preventing to use such languages for practical interests} \cite{baader2017}. On the other hand, the axiomatic restrictions of the CWA lead to decidable evaluations of DL languages \cite{PatelSch12}, albeit still intractable in some cases. Within the CWA assumption where the \textsc{Data} layer is the domain of the interpretation for both individuals and roles, and assuming that each individual is an actual object of the \textsc{Model}$\index{M}$ level, $C(i)$ could be interpreted as $\abstr(i)=C$. Please also note that, while MOF allows the representation of a multigraph, thus allowing multiple edges between two vertices, the description logic does not permit to refer to one specific edge among a given source and destination. This last observation makes the standard DL characterization unserviceable on top of our data model, thus requiring to extend such language.

Still within the field of Modal Logics but beyond DL languages\footnote{DL could be considered as a class of languages providing extensions to modal logic.}, the Register Logic language \cite{Barcelo2013} makes  constraint checking tractable at the price of the loss of expressive power of graph navigation, but allowing to execute queries always in data polynomial time (\textsc{NLogspace}).



\subsection{Ontology Alignments and Data Integration}\label{subsec:ontaling}
After describing an ontology, we're going to use ontology alignments for attacking the data integration problem. Such class of solutions have received more research acclaim than schema related one \cite{Magnani09,Magnani2010}, which are strictly representation dependent.


At this point we must extend the Description Logic syntax so that individual transcoding functions $\stigma$ are allowed, in order to be able to express supertypes'\index{supertype} transcoding functions. Such transformations will be required in the following scenario in order to say that  $C(i)\sqsubseteq C'(\transcoding(i))$. 


After describing in the previous sections how ontological alignments are useful within the process of data integration, we can now formally define what an alignment between two ontology is. Moreover, since in some scenarios the alignment uses the data representation to infer such alignment \cite{Aligon201520}, I extend the usual alignment definition provided for either schema or ontology alignments \cite{euzenat2013d,GrossHKR11} as follows:


\begin{definition}[Ontological Alignment]\label{def:ontolalignment}
An \textbf{ontological alignment}\index{alignment|textbf}  $\alignment(O,O')$ of a source ontology $O$ towards a destination ontology $O'$ is a set of tuples called \textbf{correspondences}\index{correspondence|textbf}. Each correspondence  
maps a set of types $\delta t$ from a model $M_O$ into one type $t$ in $M_{O'}$ using a transcoding \index{transcoding} function $\stigma_{\delta t\to t}$.
Such correspondence is expressed as $(\delta t,F,t,\transcoding_{\delta t\to t}, s)$, where $F$ is the alignment expressed in description logic axioms between the correspondent types and, 
whenever $\transcoding_{\delta t\to t}$ is a bijection\footnote{If $t$ expresses a \textbf{part-of} or an \textbf{is-a} relation such as $\delta t\sqsubseteq t$, then sometimes is not possible to unambiguously associate to the aggregation the single disaggregated components.}, the inverse function $\transcoding_{ t\to \delta t} = \transcoding_{\delta t\to t}^{-1}$ is also provided. An uncertainty score $s$ could be also associated to the accuracy of the alignment \cite{euzenat2013d,HartungGR13}.
\end{definition}

As a consequence of this definition, given that the ontology subsumes both the informations from the \textsc{Model}$\index{M}$ (it describes the concepts and roles) and the ones from the \textsc{MetaModel}$\index{MM}$ (TBox expressions), we have that a  \textsc{Language} $\langg_\metamodel$ could be also described by the language expressed in its correspondent ontology.
 In the case of Description Logic Ontologies, the query  \textsc{Language} is the Description Logic itself. Given that both ontologies (containing the schema definition, $\metamodel$) and queries could always be expressed within the Description Logic (\cite[Chapter~16]{Baader2010}), from now on we'll always use the terms \textit{query}\index{query}, \textit{ontology}\index{ontology} and \textit{schema}\index{schema} interchangeably as already did in \cite{Lenzerini02}.

As outlined by both more recent literature \cite{GrossHKR11,HartungGR13} and  in the former Example, it is possible to provide alignments between multiple local ontologies into one single global ontology, which in this case it is referred as \textit{hub (ontology)}. Therefore, the previous definition could be extended as follows:

\begin{definition}[Multisource Data Integration System]
	\index{data integration!multisource}
	A \textbf{multisource data integration system} is defined as a triplet $\Braket{H,\mathcal{I},\mathcal{O}}$, where $H$$\index{H|textbf}$ is the \textit{hub} ontology\index{ontology!hub|see{$H$}}, representing the target of all the  ontological alignments  $A(O,H)\in\mathcal{I}$ having $O\in\mathcal{O}$ as a source (or local) ontology.
\end{definition}
