\chapter{Data integration: a data representation-independent approach}\label{cha:dataintegration}

\epigraph{\textit{
		Logic is the most useful tool of all the arts. Without it no science can be fully known. It is not worn out by repeated use, after the manner of material tools, but rather admits of continual growth through the diligent exercise of any other science. For just as a mechanic who lacks a complete knowledge of his tool gains a fuller [knowledge] by using it, so one who is educated in the firm principles of logic, while he painstakingly devotes his labor to the other sciences, acquires at the same time a greater skill at this art.
	}}{--- William of Ockham, \textit{Summa Logic\ae}, Prefatory Letter}

This chapter describes a general approach to integrate structured, unstructured and semistructured data, while the definition of such data representations will be only provided in Chapter \vref{cha:datadef}. Theory helps in this process, since it abstracts from specific implementations. Moreover, an \textit{ad hoc} formalization allows to restrict the discussion only to the relevant properties. Abstraction is a powerful modelling language, (e.g., UML) that could later on help  to provide a final implementation. While languages and models outlining data abstraction have been defined decades ago and could be dated back to the definition to the formalization of programming languages \cite{omg96,TPLPierce}, the theoretical approaches abstracting data integration processes are quite recent \cite{Lenzerini02,DeGiacomo2018}. This lack of formalization made all the researcher focus more on the data integration aspect that are representation-dependent (\textit{how shall we integrate data with different ``shapes''}) than on the actual integration process (\textit{what makes data from different sources ``hard'' to integrate, independently from their representation?}). We can define the problem we want to solve as follows:

\begin{definition}[Data Integration]
	
  The \textbf{Data Integration} \index{data integration|textbf}is a sequence of transformations (and hence queries) through which all the data coming from different sources appear as instantiated in a uniform data structure (\textbf{reconciled representation}). The data contained in such reconciled representation is such that each datum originally coming from a data source has a value representation (\textbf{type}) \index{type!zzzza@\igobble |seealso {M}} which is indistinguishible from a datum coming from another data source (\textbf{uniform representation}).
\end{definition}

After providing some preliminary introductory examples (Section \ref{sec:oldschooldi}) showing that multiple ways to integrate different possible data representations into one most general one could be generalized by using just one approach, we will take stock of two different ways for performing integration (or fusion\footnote{In current literature the term ``fusion'' assumes an ambiguous connotation: while anglophone literature \cite{Hall97,KHALEGHI201328} focuses on the integration of the integration of sensor unstructured data possibly with structured data (e.g. geospatial information), germanophone literature \cite{deII,Bleiholder09} calls ``data fusion'' (\textit{Datenfusion}) the process of data cleaning and conflict resolution that could be carried out within one same data source. This terminological ambiguity reveals two different aspect of performing data integration.}) which are ``In-Database integration'' (Section \ref{sec:informationsintegration}) and ``Multidatabase integration'' (Section \ref{sec:integsurvey}).


\begin{figure}[pt]
\begin{minipage}[t]{\textwidth}
  \centering
  \begin{minipage}[t]{\textwidth}
    \resizebox{\textwidth}{!}{%
    \begin{tabular}{llccccl}
      \toprule
      \textbf{SocialSecurityNo} &  \textbf{diagnosis} &  \textbf{week} &  \textbf{month} & \textbf{year} &  \textbf{ward} &  \textbf{ICD-9-CM} \cr
        \midrule
      BCDVHZ59S23F743S & Right parotid neoplastic formation & 1 & 1 & 2017 & Oncology &  210.2\\
      PNPMZZ74H45H782P & Relapsing epistaxis & 2 & 1 & 2017 & Emergency &  784.7\\
      PKTBMF36E14H842O & Septal deviation and nasal-sinus polyps & 3 & 1 & 2017 & Emergency & 748.1, 471.0\\
      \bottomrule
    \end{tabular}}
    \subcaption{\texttt{Admissions} table from the internal Data Warehouse.}
    \label{tab:Admissions}
  \end{minipage}

  \begin{minipage}[t]{\textwidth}
    \resizebox{\textwidth}{!}{%
    \begin{tabular}{lllcccl}
      \toprule
      \textbf{SocialSecurityNo} & \textbf{organ} & \textbf{disease} & \textbf{day} & \textbf{month} & \textbf{year} & \textbf{ward} \cr
        \midrule
      BCDVHZ59S23F743S & \texttt{NULL} & Right vestibular deficit & 2 & 1 & 2017 & Emergency\cr % 386.10
      PNPMZZ74H45H782P & Appendix & Severe Appendicitis & 17 & 2 & 2017 & Emergency\cr % 540.9
      PKTBMF36E14H842O & Intestine & Recanalization of Crohn's Disease & 25 & 3 & 2017 & Emergency\cr % 555.0
      \bottomrule
    \end{tabular}}
    \subcaption{\texttt{Hospitalization} table from an external Data Warehouse.}
    \label{tab:Hospitalization}
  \end{minipage}

  \begin{minipage}[t]{\textwidth}
    \resizebox{\textwidth}{!}{%
    \begin{tabular}{llccccl}
      \toprule
      \textbf{SocialSecurityNo} &  \textbf{diagnosis} &  \textbf{week} &  \textbf{month} & \textbf{year} &  \textbf{ward} &  \textbf{ICD-9-CM} \cr
        \midrule
      BCDVHZ59S23F743S & Right parotid neoplastic formation & 1 & 1 & 2017 & Oncology &  210.2\\
      BCDVHZ59S23F743S & Right vestibular deficit & 2 & 1 & 2017 & Emergency & 386.10\cr
      PNPMZZ74H45H782P & Relapsing epistaxis & 2 & 1 & 2017 & Emergency &  784.7\\
      PNPMZZ74H45H782P & Severe Appendicitis & 8 & 2 & 2017 & Emergency & 540.9\cr %
      PKTBMF36E14H842O & Septal deviation and nasal-sinus polyps & 3 & 1 & 2017 & Emergency & 748.1, 471.0\\
      PKTBMF36E14H842O & Recanalization of Crohn's Disease & 15 & 3 & 2017 & Emergency & 555.0\cr %
      \bottomrule
    \end{tabular}}
    \subcaption{Expected result for the integration of the two upper tables.}
    \label{tab:MergedTables}
  \end{minipage}

  \caption{Integrating two tables (\subref{tab:Admissions} and \subref{tab:Hospitalization}) pertaining to hospitalization into one final table (\subref{tab:MergedTables}), matching with the schema of \subref{tab:Hospitalization}. Diagnoses and diseases obtaned from the ``ICD code it'' dataset, available at \url{smartdata.cs.unibo.it}, while the other parts are randomly generated.\hrule}
  \label{fig:examplesrelational}
\end{minipage}
% ---- %%%%%%%
\begin{minipage}[t]{\textwidth}
	\vspace*{1cm}
  \begin{minipage}[t]{\textwidth}
    \centering
    \includegraphics[width=.8\textwidth]{fig/01dataint/AlignmentRelational01.pdf}
    \subcaption{Alignment between the schemas of the two data sources. Data types associated to each field are not showed.}
    \label{fig:relalign}
  \end{minipage}

    \begin{minipage}[t]{\textwidth}
      \resizebox{\textwidth}{!}{%
      \begin{tabular}{lllcccl}
        \toprule
        \textbf{SocialSecurityNo} & \textbf{disease} & \textbf{week} & \textbf{month} & \textbf{year} & \textbf{ward} & \textbf{ICD-9-CM} \cr
          \midrule
        BCDVHZ59S23F743S & Right vestibular deficit & $\stigma(2,1)$ & 1 & 2017 & Emergency & $\stigma'($Rigth\dots $)$\cr % 386.10
        PNPMZZ74H45H782P & Severe Appendicitis & $\stigma(17,2)$ & 2 & 2017 & Emergency & $\stigma'($Severe\dots $)$\cr % 540.9
        PKTBMF36E14H842O & Recanalization of Crohn's Disease & $\stigma(25,1)$ & 3 & 2017 & Emergency & $\stigma'($Recanalization\dots $)$\cr % 555.0
        \bottomrule
      \end{tabular}}
      \subcaption{Record transformation for \texttt{Hospitalization} after the alignment with \texttt{Admissions}.}
      \label{tab:HospitalizationAlign}
    \end{minipage}

    \begin{minipage}[t]{\textwidth}
      \resizebox{\textwidth}{!}{%
      \begin{tabular}{lllcccl}
        \toprule
        \textbf{SocialSecurityNo} & \textbf{disease} & \textbf{week} & \textbf{month} & \textbf{year} & \textbf{ward} & \textbf{ICD-9-CM} \cr
          \midrule
        BCDVHZ59S23F743S & Right vestibular deficit & 2 & 1 & 2017 & Emergency & 386.10\cr %
        PNPMZZ74H45H782P & Severe Appendicitis & 8 & 2 & 2017 & Emergency & 540.9\cr %
        PKTBMF36E14H842O & Recanalization of Crohn's Disease & 15 & 3 & 2017 & Emergency & 555.0\cr %
        \bottomrule
      \end{tabular}}
      \subcaption{Resoluton of the  transcodng functions $\stigma$ over the aligned \texttt{Hospitalization}}
      \label{tab:HospitalizationTransf}
    \end{minipage}

    \caption{Data integration steps: intermediate schema alignment and transcoding transformation steps before providing the final result.}
\end{minipage}
\end{figure}
The outline of the generic process required for  data integration will be outlined  in Section \vref{sss:gdi} as a generalization of the data integration process that are here described.
