\section{Conclusions}
After showing data integration over specific data models, we discussed a general strategy abstracting from particular data representations. We showed that current data models are not able to express structural aggregation, where coarse data representation and finer ones cannot coexist within one single instance. We also showed that current query languages (e.g., SQL and SROIQ) fail at representing either aggregations or alignment tasks. As a consequence, data integration requires both a generalised data model providing the desired structural aggregation and a query language (over such general representation) expressing queries currently used for specific types of data sources. As we're going to see in Chapter \ref{cha:NGQL}, our proposed query language is  able to express the $\Qoppa$, $\alpha$ and $\nu_\cong$ operators for data integration, that can now be only supported outside traditional query languages. All these concepts are going to be addressed and solved within this thesis, either from a formal point of view, or on an algorithmic one ($\nu$ and $\bowtie$). With respect to graph data, the present thesis is going to show that it is possible to provide efficient implementations of both graph joins (Chapter \ref{cha:join}) and graph nesting (Chapter \ref{cha:nesting}) operators.

The analysis of MetaObject Facility data model showed that query languages should be a part of the desired data model: in particular, a subset of such query languages asserting data properties or transforming data representations should be representable within the data model. Chapter \ref{cha:graphsdef} (with special reference to Section \vref{sec:scriptEll}) will show that this feature allows a straightforward characterisation of structural aggregations within our Generalized Semistructured data Model.
