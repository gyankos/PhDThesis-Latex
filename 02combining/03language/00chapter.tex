\chapter{GSQL: a Generalized Semistructured Query Language}\label{cha:NGQL}
\epigraph{Language is a process of free creation; its laws and principles are fixed, but the manner in which the principles of generation are used is free and infinitely varied. Even the interpretation and use of words involves a process of free creation.}{--- Noam Chomsky, \textit{Language and Freedom}, (87-8)}


The definition of a new data model requires a new  query language: even though several distinct algorithms have been developed distinctly for graph and semistructured data integration, the definition of an algebra (and hence a set of operations) can detect which are the common steps in both scenarios, so that they can be expressed through a common subset of minimal operators. %A similar approach has been already proposed in data mining  within the relational model \cite{Calders2006}. Therefore, \hl{this chapter  presents the paNGRAm language for nested graphs}, which is composed by two other language building blocks: first, \texttt{script} defines the  language for GSMs' $\lang$ expressions defining expressions within the GSM model and provides a syntax in which both predicates and object transformation functions are possible. Second, 
The proposed \textsc{General Semistructured Query Lanugage} may also use the \texttt{script} expressions introduced in the previous chapter to provide predicates and functions ($\langg_{\metamodel}\equiv \textup{GSQL}_{\texttt{script}}$), thus providing the basic building block operations over which we will then evaluate semistructured path queries, graph traversals, implement the (semistructured) relational algebra (for data mining) and the nested graph operators.

Moreover, this chapter will also show that traditional traversal languages may be also  expressed via GSQL operators such that GSM are closed over the GSQL expressions. By showing that both filtering queries and algebraic operators may be expressed within the same language, it would be also possible to perform mutual optimizations between such operators: this intuition will lead into the next chapter, where it will be showed how graph pattern matching queries can be nested within graph nesting operators. This final scenario will motivate the need of such a low level algebra for describing each possible optimizable step, thus leading towards some more future works on higher level operation over GSMs.


