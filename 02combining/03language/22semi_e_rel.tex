%%%%%%%%%%%%%%%%%%%%%%%%%%%%%%%%%%%%
%%%%%%%%%%%%%%%%%%%%%%%%%%%%%%%%%%%%
%%%%%%%%%%%%%%%%%%%%%%%%%%%%%%%%%%%%
\subsection{Relational and semistructured operations}\label{ssec:gsmrelop}
 The previous subsection discussed the definition of set (and object) operators on top of GSQL. \marginnote{$\qedsymbol$ \textit{Attribute labelled Set operators may be used over relations' GSM representations.}} For this reason, we must make sure that the former definitions are also compliant with GSM translations of relations as defined in Algorithm \vref{alg:reltonested}: given that the aforementioned translation maps each possible relation into one object $r_o$ associating all the tuples $t_i$ via a single \ONTA attribute ($t_i\in \phi(r_o,\ONTA)$), by definition of the former operators we have that the desired set operation will be actually performed over the set of objects defined over \ONTA. Therefore, the former section provides a straightforward definition for set operations that are compliant with objects produced by a translation from the relational model, and hence they do also implement set operations for the relational model. Moreover, the union actually defines an outer union \cite{deII}, where the resulting relation has the union of  two relations' schemas.

Similar considerations can be also performed for the relational filtering (also known as selection, $\sigma$) operator: if we restrict the filtering property just to the elements contained by the object reference in \ONTA, we can express the $\sigma_P$ operator from $\texttt{filter}_P$ presented in Equation \vref{eq:selection} as follows:
\[\texttt{filter}_{\scriptline{t -> \{not (t in (g.phi[}\ONTA\texttt{])) {\color{RoyalBlue}||} } P\texttt{(t)\}}}(n)\]
Please note that such restriction may be completely ignored for semistructured models, where the \texttt{filter} operator may be directly introduced.


At this point, we want to show that the $\texttt{map}$ operator can express other algebraic operators that have  already been defined in current literature, such as  embedding ($\varepsilon_{EF}$) and the projection ($\pi_{PF}$) operators presented in \cite{Magnani06}. Both operations acts as a specific instance of map operators: while $EF$ is defined as a function extending the object's collection with new identifiers or creating new associated collections, $PF$ either reduces the number of collections or reduces their content. Consequently, both operators can be defined as follows:

\begin{definition}[Embedding]
	Given a GSM $n$, its \textbf{embedding}\index{GSQL!$\varepsilon$} is defined as a specific map function $EF$ such that $\forall o\in O.\forall p\in\lang. EF(o,p)\supseteq \phi(o,p)$:
	\[\varepsilon_{EF}(n)=\texttt{map}_{\ell_{nn'},\xi_{nn'},\phi_{nn'}\oplus EF}(n)\]
\end{definition}

\begin{definition}[Projection]
	Given a GSM $n$, its \textbf{projection}\index{GSQL!$\pi$} is defined as a specific map function $PF$ such that $\forall o\in O.\forall p\in\lang. EF(o,p)\subseteq \phi(o,p)$:
	\[\pi_{PF}(n)=\texttt{map}_{\ell_{nn'},\xi_{nn'}, EF}(n)\]
\end{definition}

In order to express $\pi$ as the one defined within the relational model, we can simply express $PF$ as the relevant $\phi$ attributes that must be returned, instead of specifying the whole $PF$ definition. If we want to explicitly create new objects resulting from an expression evaluation $f$ returning the list of values to be associated via $\xi$, we can define the $Calc$ operator \cite{Calders2006} as follows:


\begin{definition}[Calc]\label{def:calc}
\index{GSQL!$Calc$}
Given a GSM $n$, the $Calc$ operator extends each object $x$ appearing in $\phi(\ngraph,K_1)$ with a newly created object $(o+1)_c$ contained in $\phi(o,K_2)$; $(o+1)_c$ will have a  label set $A$ and value $f(x)$:
\[Calc_{f\texttt{ as A}}^{K_1,K_2}(n)=\texttt{fold}_{\jsem{\texttt{o.\textbf{phi}[K]}}_{\{(\texttt{g},\ngraph),(\texttt{K},K_1)\}},f_K}(n)\]
where $f\colon O\mapsto\partof{\metamodel}$ and $f_K$ is the accumulation function which is defined as follows:
\[f_K=x\mapsto \alpha\mapsto \textbf{let }o_c\eqdef\max O\textbf{ in }\texttt{map}_{\jsem{\texttt{o.\textbf{ell}}}_{\emptyset}\,,\,\jsem{\texttt{o.\textbf{xi}}}_{\emptyset}\,,\,\jsem{\texttt{e}}_{(\texttt{oc},(o+1)_c),(\texttt{K},K_2),(\texttt{x},x)}}\left(\texttt{create}^{{(o+1)}_c}_{A,f(x),\emptyset}(\alpha)\right)\]
and \texttt{e} is the expression performing the $K_2$ extension for each $x$ as follows:
\begin{center}
\texttt{map(o.\textbf{phi} : z $\mapsto$ \{if (o.\textbf{id} == x.\textbf{id} \&\& z[0] == K) then \{z[0], z[1] @ \{oc\}\} else z\})}
\end{center}

\end{definition}
If $n$ is a relation obtained using the canonical transformation (Algorithm \vref{alg:reltonested}) where each relation contains its tuples in $\phi(\ngraph,\ONTA)$ and each tuple in is represented by an object $o$ containing its attribute in  $\phi(\ngraph,\mstr{Attribute})$, we have that $K_1\eqdef\ONTA$ and that $K_2\eqdef \mstr{Attribute}$.

\marginnote{\textit{Expressing the class of nesting operations.}}In Section \vref{sec:informationsintegration} we observed that a nesting operator (Equation \vref{def:nestingfirst}) can be used to generalize both joins and grouping operations. We want to generalize the already-existing join and grouping operations by allowing to express the broader class of clustering algorithms which are both overlapping and partial \cite{Tan05}, thus allowing to express the social network clustering. In order to define such class of operations, we have to define the object classifier  $GF$ as a function mapping each object into a (possibly empty) subset of classes in $\mathcal{C}$ ($GF\colon O\mapsto \partof{\mathcal{C}}$). As showed in the part-of aggregation example (see Example \vref{ex:partof}), the outcome of such  can be then summarized into one single object: therefore, the desired operation shall generalize most of the algebraic grouping operations.


This operation requires an expression $\oplus_f$ (Equation \vref{eq:nestingAggregation}) providing  an aggregation over either similar or equivalent elements, and a way to generate collection of collections from a initial collection: this last step must use the aforementioned clustering operation $GF$, as outlined in \cite{Magnani06}. The definition of such  operator allows to group all the elements belonging to the same cluster and leaves out all the non-represented outliers, that may be included or not in the final result. 
\medskip

%In particular, the broadest class of clustering algorithm are both overlapping and partial. This means that such algorithms act as a classifier function $GF$,, while allowing to express some real use case scenarios that cannot be expressed with current operators. 

As previously discussed for both the $\varepsilon$ and $Calc$ operators, this data model does not allow to refer to elements that still do not exist. For this reason each element expressing the result of an aggregation must be created before effectively aggregating the desired components. Moreover, in order to create multiple elements, we must  iterate over the all possible clusters minable within each object's collection, and then detect which are the group to be created. The set over which the iteration is going to be performed is defined as follows:
% do actually allow the creation of such elements. Therefore, the aggregation operation can be expressed similarly to the link discovery operator with an additional mapping operator, which replaces the aggregated elements with the result of their aggregation. In particular, the set over which perform the $fold$ can be defined as follows:
\[S_{GF,n}\eqdef\Set{(o,p,k)\Big|o\in \varphi(n),p\in\dom(\phi(o)),\phi(o,p)\neq\emptyset, k\in\bigcup_{o'\phi(o,p)} GF(o,p,o')}\]
Each triplet $(o,p,k)$ contained in this set associates to each non-empty collection $\phi(o,p)$ a labelled cluster $k\in\mathcal{C}$ containing at least one element of $\phi(o,p)$. Therefore, all the elements matched by with $k$ in $\phi(o,p)$ are going to be replaced by one single object. This new object has to be created by the $\oplus_f$ function mentioned in the third chapter as follows: 
%This set associates to each containing object $o$ within an attribute $p$ one cluster $k\in\mathcal{C}$ representing at least one element in $\phi(o,p)$ which, consequently, must be non empty. Therefore, each triplet must be associated to the creation of such final cluster elements via $\oplus_f$ on the previous state $\alpha$ of computation as follows:
\[\oplus_f(\alpha,k,\omega,\Set{o'\in\phi(o,p)|k\in GF(o,p,o')})\]
where $\alpha$ is a previous step of $n$ where another object pointed by $S_{GF,n}$ was generated, and $\omega$ is the object $id$ associated to the object generated by the expression associated to $\oplus_f$. In particular we can arbitrarily choose to set $\omega$ using an $id$ generation function $gen(o_{\tilde{c}},k,o,p)$, where\footnote{Some use cases that will follow are going to use the following definition of $gen$: \[\genp(o_{\tilde{c}},k,o,p)=\Set{(o+dtl([\texttt{bin}(k),o,\texttt{bin}(p)]))_{\tilde{c}+1}}\] where \texttt{bin} is the function associating to each element its byte representation expressed as \texttt{bigint} compatible with the $id$ definition. By using the dovetailing function over the binary representation of the triplet, we ensure that different $id$-s are going to be associated to their correspondent generated objects.} $o_{\tilde{c}} = \max n.O$. The newly created object by $\oplus_f$  can be concatenated to the object generator via $S_{GF,n}$ via a fold iterator as follows:
\begin{equation}\label{eq:ce}
ce\eqdef=\texttt{fold}_{S_{GF,n},(o,p,k)\mapsto \alpha\mapsto\textbf{let }o_{\tilde{c}}=\max O\textbf{ in } \oplus_f(\alpha,k,gen(o_{\tilde{c}},k,o,p),\Set{o'\in\phi(o,p)|k\in GF(o,p,o')})}(n)
\end{equation}
where $n$ provides the initialization of the accumulator for the fold operation.
%Now, we can create the aggregated elements as follows: for each non-empty $\phi(o,p)$ we associate a new aggregated object if it contains at least one element belonging to a class in $\mathcal{C}$:
%\[\textup{fold}_{S_{GF,n},f_1}(n)\]
%where the function $f_1$ is the function that creates a new aggregated object with id $(\max O+dtl([\texttt{bin}(c),o,\texttt{bin}(p)]))_{c+1}$, label $c$ and containing all the elements in $\phi(o,p)$ belonging to the class $c$:
%\[((o,p,c),a)\mapsto \kappa^{\max O+dtl([\texttt{bin}(c),o,\texttt{bin}(p)])_{c+1}}_{\Set{c},\emptyset,\texttt{bin}(c)\mapsto [o'\in\phi(o,p)|c\in GF(o)]}(a)\]
After creating the objects associated to the mined clusters, we can now replace the objects of $\phi(o,p)$ belonging to a cluster $k$ by using a map, which retrieves the clustered objects via $gen$. All those intermediate computational steps may be chained together into the following definition of a nesting operator:

\begin{definition}[Nesting]\label{def:semistructnest}
	Given a GSM $\fullnested$, the \textbf{nesting}\index{GSQL!$\nu$} operator $\nu_{GF,gen,\oplus_f}^{\textbf{keep}}$ aggregates the elements within each non empty collection $\phi(o,p)$ (where $o\in O$ and $p\in\dom(\phi(o))$) by replacing with one single object all the elements belonging to the same class $k\in \cod(GF)$. Moreover, all the elements that belong to no class are not aggregated. The outliers may be returned (\textbf{keep}\texttt{=tt}) or not (\textbf{keep}\texttt{=ff}), dependingly on the desired final representation. The operator is defined as follows:
	\[\hspace{-.25cm}\nu^{\textbf{keep}}_{GF,gen,\oplus_f}(n)=\texttt{map}_{\ell,\xi,o\mapsto p\mapsto [o'\in \phi(o,p)|GF(o,p,o')=\emptyset\wedge \textbf{keep}]\;\cup\;\bigcup[gen(o_{\tilde{c}},k,o,p)|k\in\bigcup_{(o,p,o')\in\dom(GF)} GF(o,p,o')]}(ce)\]
where ``$ce$'' was defined in Equation \vref{eq:ce} and $o_{\tilde{c}} = \max n.O$. Please note that this operation uses the $gen$ function to associate the associations and classes generated by $GF$ to the objects that are generated in the ``$ce$'' phase.
	%where $f_2$ is defined as the function changing the nesting content of $\phi(o,p)$ with all the non aggregated elements (appearing on the right square) with the previously aggregated ones (on the right):
	%\[f_2\eqdef o\mapsto p\mapsto \Big[o'\in\phi(o,p)\Big|GF(o')=\emptyset\Big]\cup\Big[(\max O+dtl([\texttt{bin}(k),o,\texttt{bin}(p)]))_{c+1}\Big|k\in\cup GF(\phi(o,p))\Big]\]
\end{definition}

Given that this operation can replace any object within any collection, $GF$ can be constrained within the relational model by ensuring that $GF$ must return an empty set for any $(o,p,k)$ where $o$ does not appear as an entity within $n$. This approach is similar to what it has been previously stated for \texttt{filter}. The former definition (and restrictions) allows to instantiate allowing to obtain the other derived operators, for both relational and semistructured models. 


Before introducing all the possible derivations for such operator, we weant to introduce the last remaining operator, which is the opposite operation of nesting, that is the unnesting operation. In this case, we must select which element $o'\in\phi(o,p)$, within a given object $o$ and associated to an attribute $p$, has to be replaced by its expansion in $\phi(o,p)$, over a set of given attributes $A$. In particular, we're going to select which elements are going to be expanded, dependingly to the attribute $p$ where $o'$ is contained and on $o'$ itself.

\begin{definition}[Unnesting]
	Given a GSM $n$, a set of attributes $a\in A$ -- over which replace and expand via the objects $o'$ ($\phi(o',a)$) that are contained in $o'\in\phi(o,p)$ --, and a binary predicate $P$ through which select the $o'$ appearing in $p$ ($P(p,o')$), the \textbf{unnesting}\index{GSQL!$\mu$}  operator is defined as follows:
	\[\mu_{A,P}(n)=\texttt{map}_{\ell,\xi,o\mapsto p\mapsto [o'\in\phi(o,p)|\neg P(p,o')]\;\cup\;\bigcup[\phi(o',p')|p'\in A, o'\in\phi(o,p), P(p,o')]}(n)\]
\end{definition}

As we will see in the following four operators, relational joins, grouping and abstraction operators may be all derived from the nesting operator.

\phparagraph{Data-Preserving Aggregation ($\alpha_2$)}
We can generalize the aggregation operator by associating all the $GF$-similar elements to one single element, containing all the references. In particular, we use the $\genp$ function for generating new ids, because each newly generated id must belong to the containment $p$ of a specific object $o$, and shall not appear in other containments for different $o$-s and $p$-s. In order to do so, we let the user decide on how to represent the resulting set of labels, the set of expressions, and the containment respectively through $f_L$, $f_E$ and $f_C$ functions. \index{GSQL!$\alpha_2$}
\[{\alpha_2}_{GF,f_L,f_E,f_C}^{\textbf{keep}}(n)=\nu^{\textbf{keep}}_{GF,\genp,(\alpha,k,\omega,s)\mapsto \texttt{create}^\omega_{f_L(k,s),f_E(k,s),f_C(k, s)}(\alpha)}(n)\]



\begin{figure}[!ph]
\centering
\begin{minipage}[t]{0.9\textwidth}
	\centering
	\includegraphics[scale=0.8]{fig/05language/10rearranged_input.pdf}
	\subcaption{Rephrasing the data input for a bibliographical network represented in Figure \vref{subfig:transformationandmatch}.}
	\label{fig:nestingexamples}
\end{minipage}
\begin{minipage}[t]{0.9\textwidth}
	\centering
	\includegraphics[scale=0.8]{fig/05language/11preserving_aggregation.pdf}
	\subcaption{${\alpha_2}_{GF_1,(k,s)\mapsto [k], (k,s)\mapsto \emptyset, (k,s)\mapsto[[k,\, s]]}^{\texttt{\color{RoyalBlue}tt}}(n)$}
	\label{fig:firstexample}
\end{minipage}
\begin{minipage}[t]{0.9\textwidth}
	\centering
	\includegraphics[scale=0.8]{fig/05language/12_intermediate_aggregation.pdf}
	\subcaption{$\texttt{map}_{x\mapsto \ell(x)\backslash{\hstr{\alpha}},\xi,\phi}(\gamma_{GF_2}^{\texttt{\color{RoyalBlue}tt}}({\alpha_2}_{GF_1,(k,s)\mapsto [k,\hstr{\alpha}], (k,s)\mapsto \emptyset, (k,s)\mapsto[[k,\, s]]}^{\texttt{\color{RoyalBlue}tt}}(n)))$}
	\label{fig:secondexample}
\end{minipage}
\caption{Representing different possible results from the application of the general nesting definition. \textit{(cont.)}}
\label{fig:nestingoperations}
\end{figure}

\begin{example}[label=ex:aggregations]
In Example \vref{ex:examplegraphdata} we addressed the problem of extracting the schema from a JSON representation of a graph. As we previously outlined, we could choose to use an aggregation where each matched component via a class $k$ is nested within an object $o$ via $k$, and that $k$ is used as a label for the object that will contain such data. Therefore, the desired result can be achieved via the following application of the $\alpha_2$ aggregator:
\[{\alpha_2}_{GF_1,(k,s)\mapsto [k], (k,s)\mapsto f(s), (k,s)\mapsto[[k,\, s]]}^{\texttt{\color{RoyalBlue}tt}}(n)\]
At this point we want to aggregate each object by its associated label; if the object is a \mstr{label} object, we want to return the value associated to it designign the containing object's label; if the object ha associated to a non relevant label w.r.t. the schema extraction process (e.g. \mstr{metadata}), a set containing the concatenation of the $GF$ classes of all the concatenating object is returned; in all the other cases, no nesting is performed. the following clustering operation describes the desired result:
\begin{equation}
\label{eq:GF1}
GF_1(\_, \_, o)=\begin{cases}
\emptyset & \mstr{metadata}\in\ell(o)\vee P(o)\\
\xi(o) & \mstr{label}\in\ell(o)\\
%\xi(o') &  \min_n (o'\in\varphi^n(o) \wedge \ell(o')=\mstr{label})\\
%\{\bigodot GF(\varphi(o))\} & \mstr{metadata}\in\ell(o)\\%\vee \exists o'\in\varphi(o). GF_1(o')\neq\emptyset\\
\ell(o) & \ell(o)\neq \emptyset\\
\emptyset & \textup{oth.}\\
\end{cases}
\end{equation}
where the underscores remark the ignored arguments.  $P$ is a predicate avoiding to aggregate the elements that are represented only once within the hierarchy and appear at the coarsest levels of it; such predicate is defined as follows:
\[P(x)=(\ell(x)=\emptyset\vee (\neg\exists o'\in\varphi^*(o.n). o'\neq x\wedge \ell(o')=\ell(x)))\wedge(\forall o'\in\varphi^*(o.n).x\in\varphi^+(o')\Rightarrow P(o'))\]
The result of the application of such aggregation to the nested structure represented in Figure \ref{fig:nestingexamples} is provided in Figure \ref{fig:firstexample}: this operation preserves all the original data within each aggregated element but, at the same time, increases the amount of generated data. As a consequence, this solution considerably increases the time required to visit the data structure. Therefore, a different approach preserving the GSM height in spite of the representation of the original information is required.
\end{example}

\phparagraph{Grouping ($\gamma$)}
The grouping operation for semistructured data  was originally presented in \cite{Magnani06}, where  $\oplus_f$ is simply defined as the $n$-ary union of all the matched objects, thus allowing to integrate each similar component into one single representation. The desired operation can be described as follows: \index{GSQL!$\gamma$}
\begin{equation}\label{eq:grouping}
\gamma_{GF}^{\textbf{keep}}(n)=\nu^{\textbf{keep}}_{GF,\genp,(\alpha,k,\tilde{o}_{c},s)\mapsto \texttt{elect}_{\ngraph}\left(\bigcup^{\tilde{o}_c}_{i\in s}\texttt{elect}_i(\alpha)\right)}(n)
\end{equation}
\begin{example}[continues=ex:aggregations,label=ex:aggregations2]
Figure \ref{fig:secondexample} shows an example of grouping. In this case we want to aggregate even the elements that were not previously inserted within a cluster. Therefore, we change the $\alpha_2$ by marking with $\hstr{\alpha}$ the elements matched by $GF_1$.
%\begin{equation}
%\widetilde{GF_1}(\_,\_,o)=\begin{cases}
%\emptyset & P(o)\\
%\xi(o)\cup\{\hstr{\alpha}\} & \mstr{label}\in\ell(o)\\
%\{\bigodot GF(\varphi(o))\}\cup\{\hstr{\alpha}\} & \mstr{metadata}\in\ell(o)\\
%\ell(o)\cup\{\hstr{\alpha}\} & \ell(o)\neq \emptyset\\
%\emptyset & \textup{oth.}\\
%\end{cases}
%\end{equation}
Last, the following $GF$ function for $\gamma$ is provided:
%\[GF_2(\_,\_,o)=\begin{cases}
%\ell(o)\backslash\{\hstr{\alpha}\} & \hstr{\alpha}\in\ell(o)\\
%\ell(o) & (P(o)\wedge \ell(o)\neq\emptyset ) \vee \exists o'\in\varphi^+(o).\hstr{\alpha}\in\ell(o') \\
%\xi(\min S) & P(o)\wedge (S= \argmin_{\tsub{o''\in \varphi^*(o.n)\cap\varphi^+(o)\\ \mstr{label}\in\ell(o'')}}rh(o,o'') \wedge S\neq\emptyset)\\
%\emptyset & \textup{oth.}\\
%\end{cases}\]
\[GF_2(\_,\_,o)=\begin{cases}
\ell(o)\backslash\{\hstr{\alpha}\} & \hstr{\alpha}\in\ell(o)\\
\ell(o) & \ell(o)\neq\emptyset \wedge \exists o'\in\varphi^+(o). GF_2(\_,\_,o')\neq\emptyset \\
\bigodot_{o'\in\varphi(o)}GF_2(\_,\_,o') & \ell(o)=\emptyset\\
\emptyset & \textup{oth.}\\
\end{cases}\]
where $\bigodot$ is the string concatenation function over set of strings.
As we can see, this operation does not structurally propagate the aggregation within all the containment levels, but it only aggregates the data at the first nesting level available. In order to propagate the nesting in depth, we must iterate the same operation until all the similar components are structurally aggregated together. Therefore, it is now relevant why the \texttt{fold} construct is relevant for our algebra.
\end{example}


\phparagraph{Abstraction ($\alpha_1$)}\label{abstractionAlpha1}

Let us now discuss on how to achieve the $\alpha$ schema extraction operator over our nested data representation. In the previous chapter we mentioned that, within this thesis, we are going to work exclusively on nesting-loop free GSMs: this constraint allows a definition of structural length of a GSM; given that the structured aggregation must be further propagated towards the leaves after each iteration. Consequently, we can consider the GSM's height as an upper bound to the number of iterations required to propagate the aggregations as expected. Therefore, we may first perform an $\alpha_2$ aggregation and, after that, we can recursively group by the former labels. Such operator may be defined as follows:



\begin{definition}[Structural Aggregation]
	\index{GSQL!$\alpha_1$}
	Given an equivalence relation $SF$ to be tested among the objects within each containment $\phi(o,p)$, the structural aggregation propagates the aggregation result to all the underlying data structures by marking them with a $\hstr{\alpha}$ label. This operator is then defined as follows:
	\[{\alpha_1}_{SF}^{\textbf{keep}}(n)=\gamma_{GF_2}^{\textbf{keep}}(\alpha)(\texttt{fold}_{\Set{i\in \nat|0<i< h(n)},x\mapsto\alpha\mapsto \gamma_{GF_2}^{{\color{RoyalBlue}\texttt{tt}}}(\alpha)}({\alpha_1}_{SF,(k,s)\mapsto [k,\hstr{\alpha}], (k,s)\mapsto \emptyset, (k,s)\mapsto[[k,\, s]]}^{{\color{RoyalBlue}\texttt{tt}}}(n)))\]
\end{definition}

\begin{figure}[!t]
	\ContinuedFloat
	\centering
	\begin{minipage}[t]{0.5\textwidth}
		\centering
		\includegraphics[scale=0.8]{fig/05language/13actual_schema.pdf}
		\subcaption{$\texttt{map}_{x\mapsto \ell(x)\backslash{\hstr{\alpha}},\xi,\phi}({\alpha_1}_{GF_1}^{{\color{RoyalBlue}\texttt{ff}}}(n))$}
		\label{fig:finalexample}
	\end{minipage}
	\caption{Representing different possible results from the application of the general nesting definition.}
\end{figure}
\begin{example}[continues=ex:aggregations2]
Figure \vref{fig:finalexample} provides the desired solution allowing to implement the schema extraction operation outlined in Example \vref{ex:examplegraphdata} for the data integration scenario.
In order to met the requirements of the former definition, $GF_1$ presented in Equation \vref{eq:GF1} is used for the first aggregation step, while the remaining ones are performed via the following function applied to has to be extended in order to mark the elements that must be structurally aggregated.
\end{example}

After performing this operation, we can now aggregate the leaft out elements, thus allowing to implement the relational group over overlapping classes.

\begin{definition}[Multi Group-By]
	\index{$\Gamma$!for GSQL|see {GSQL}}\index{GSQL!$\Gamma$}
Given an equivalence relation $SF$ to be tested among the objects within each containment $\phi(o,p)$ and an aggregation function expressed by the three functions $f_L,f_E,f_C$ to be applied over the remaining non-aggregated objects via $SF$, the \textbf{multi group by} over a GSM $n$ is defined as follows:
	\[\Gamma_{SF}^{f_L,f_E,f_C}(n)={\alpha_2}_{(o,p,o')\mapsto {\color{webgreen}\textup{``}\alpha\textup{''}}\notin\ell(o')?{\color{webgreen}\textup{``}\alpha\textup{''}}:\emptyset}^{\texttt{\color{RoyalBlue}tt},f_L,f_E,f_C}({\alpha_1}_{SF}^{{\color{RoyalBlue}\texttt{tt}}}(n))\]
\end{definition}

%\begin{description}
%\item[Aggregation:] 
%From this definition, many other definitions are derived:
%\begin{itemize}
%\item \textbf{Simple Nesting.} In the most simple and general case, we want to ``embed'' the set of objects belonging to the same cluster inside a newly created object via an attribute expressing the cluster $k$ detected from a classifier $GF$. Therefore, $\oplus_f^\omega$  is a \texttt{create}$^\omega$. The simple definition is provided as follows:
%\[\dot\nu_{GF,f}^{\textbf{keep}}(n)=\alpha_2_{GF,(k,s)\mapsto [k], (k,s)\mapsto f(s), (k,s)\mapsto[[k,\, s]]}^{\textbf{keep}}(n)\]
%Figure \texttt{[TODO]} shows an example of simple nesting. As we can see, this operation duplicates the height of the GSM, thus increasing the visiting time for navigating the data structure in depth. Therefore, we can also define another different definition as follows:
%
%
%\end{itemize}

%\item[Grouping: ] 
%Since the union operation is defined by the composition of a \texttt{disjoint} with a \texttt{map}, where the $c$ is incremented by one unit, in order to have the expected result in the general nesting's \texttt{map} operation, we must decrement the $c$ argument by one unit.
%Moreover, in order to successfully allow to aggregate together similar elements, $GF$ must return for each object $o$ the set of classes corresponding to the values that we want to be matched.

\phparagraph{$\otimes\theta$-Product}
\index{product!$\otimes_\theta$!for GSQL|see {GSQL}}\index{GSQL!$\otimes_\theta$ product}
\label{def:otimesthetaList} In this scenario we must suppose that we preliminarily merge the two operands $n$ and $n'$ via a disjoint union $\texttt{disjoint}^{\omega_c}(n, n')$. Similarly to what it has been already defined for relational and set operations, we want to consider the input GSM's reference objects as multiple labelled sets and, therefore, the final join operation shall be performed among all the collections appearing in the left and right  operand. In particular, for each collection $\phi(\omega_c,[1,p'])$ and $\phi(\omega_c,[2,p''])$ respectively coming from the first and second operand, we want to return a new collection $\phi(\omega_{c+1},p'p'')$ containing the result of the $\otimes_\theta$ product of the contained objects. As a final result, a \texttt{create} predisposing the $p'p''$ collection has to be performed immediately after the disjoint union.%, and that $GF$ contains the reference to the binary predicate $\theta$ and is defined as follows:

The clustering function $GF$ shall create a distinct cluster for each pair of matching objects $x$ and $y$, respectively contained in the collections $\phi(\omega_c,p')$ and $\phi(\omega_c,p'')$, and then associates each object to the pairs $\texttt{bin}(p')\texttt{bin}(p'')xy$. Such set of elements will be used in the clustering function (Equation \ref{positive-subnum}) and can be defined as follows:
\[\begin{split}
JS_{\omega_c,p',p''}(x)=&\{(\texttt{bin}(p')\texttt{bin}(p'')xy)_{c+1}|\theta(x,y),x\in\phi(\omega_{c},[1,p']),y\in\phi(\omega_{c},[2,p'']) \}\\
	&\cup\{(\texttt{bin}(p')\texttt{bin}(p'')yx)_{c+1}|\theta(y,x),x\in\phi(\omega_{c},[2,p'']),y\in\phi(\omega_{c},[1,p']) \}
\end{split}\]
On the other hand, each $[1,p]$ and $[2,p'']$ collection contained in $\texttt{disjoint}^\omega(n',n'')$ must be empty (Equation \ref{GFotherwise}) while all the remaining objects' containment should be kept unaltered (Equation \ref{negative-subnum}).The whole definition of $GF$ is defined as follows:
\begin{subnumcases}{GF_{\theta}(o,p,x)=}
JS_{\omega_c,p',p''}(x) & $o=\omega_{c+1},p=p'p''.[1,p'],[2,p'']\in\dom(\phi(\omega_c))$ \label{positive-subnum}\\
\bot & $o\neq\omega_{c+1}$ \label{negative-subnum}\\
\emptyset & oth.\label{GFotherwise}
\end{subnumcases}
%\[\hspace*{-1.25cm}GF_\theta(o_{\tilde{c}},p,x)=\begin{cases}
%	\Set{(\bin(p)\cdot \bin(p')\cdot x\cdot y)_{\tilde{c}}|y\in \phi(o,[2,p']), \theta(x,y)} & o_{\tilde{c}}\equiv \ngraph, p\in\dom(\phi(o)), x\in \phi(o,[1,p])\\
%	\Set{(\bin(p')\cdot\bin(p)\cdot y\cdot x)_{\tilde{c}}|y\in\phi(o,[1,p']), \theta(y,x)} & o_{\tilde{c}}\equiv \ngraph, p\in\dom(\phi(o)),x\in\phi(o,[2,p])\\
%	\big(\bigcup_{{z\in\phi(o,p')}} GF_\theta(o,p',z)\big)\cap \big(\bigcup_{{y\in\phi(o,p'')}} GF_\theta(o,p'',y)\big) & o_{\tilde{c}}\equiv \ngraph,p = p'p'', p\notin \dom(\phi(o))\\
%	\bot & o_{\tilde{c}}\nequiv \ngraph
%\end{cases}\]
This definition is then involved in two different roles: first, \textit{(\textbf{i})} $GF_\theta$ is used in Equation \vref{eq:ce} to generate the pairs $(o,p'p'',\texttt{bin}(p')\texttt{bin}(p'')xy)$, so that the aggregation function $J_\otimes$ used for $\oplus_f$ is able to generate a new object by concatenating the two  objects  $u$ and $v$ matching with predicate $\theta$ for containments $b_1$ and $b_2$:
\[J_\otimes(\alpha,\texttt{bin}(p')\texttt{bin}(p'')xy,\omega,s)=\texttt{elect}_{\omega}\left(\bigotimes^{\texttt{bin}(p')\texttt{bin}(p'')xy}_{u\in s}\texttt{elect}_u(\alpha)\right)\]
where $\otimes$ is a generic aggregation operation for each element $u$ appearing in the cluster for the elements matching the cluster label $\texttt{bin}(p')\texttt{bin}(p'')xy$ that is going to be used for an object identifier for the previously-aggregated element. Then, the operators re-set $\omega$ as a reference object over which perform the remaining operations. 
For the relational join purposes, we can choose the previously-introduced  concatenation operator $\oplus$ as $\otimes$ for combining the matched objects. 
\medskip

Last, \textit{(\textbf{ii})} the $GF_\theta$ classifier is used in cooperation with $\genp'$ to define which are the resulting containments, directly generated by $GF_\theta$ as a result of the $\otimes\theta$-product operator, and which are the elements that are not involved by the $\otimes\theta$-product operation ($o_{\tilde{c}}\neq \omega$); while in the first case the result of the $\otimes_\theta$ combination shall be returned for each matched pair of objects (Equation \ref{vargammafst}), for the other cases where the resulting reference object is not involved and hence the involved \texttt{map} must neither change nor update their containments must be preserved (Equation \ref{vargammasecond}). 
\begin{subnumcases}{\genp'(o_{\tilde{c}},k,o,p)=}
\phi(o,p) & $k = \bot$ \label{vargammafst}\\
k &  oth.\label{vargammasecond}
\end{subnumcases}
Consequently, the binary $\otimes\theta$-product operation can be defined as follows\index{product, $\otimes_\theta$!for GSM}:
\[n \otimes_\theta n'=\nu_{GF_\theta,\genp',J_\otimes}^{\texttt{\color{RoyalBlue}ff}}(\texttt{create}^{\omega_{c+1}}_{\ell(\omega_c),\xi(\omega_c),\oplus_{\tsub{$[1,p']$,\\ $[2,p'']\in\dom(\phi(\omega_c))$}}\;[[p'p'', \phi(\omega_c,[1,p'])\cup \phi(\omega_c,[2,p''])]]\;}(\texttt{disjoint}^{\omega_c}(n,n')))\]
Consequently, by replacing $\otimes$ with $\oplus$ we achieve and by constraining $\theta$ to check that all the elements contained by the to-be-merged elements and having the same $\ell$ label must show the same $\xi$ values, we have the implementation of the join operator, thus the following expression provides the join definition alongside the definition of the restriction for $\theta$:
\[n \bowtie_\theta n'\eqdef n \oplus_{\substack{(x,y)\mapsto\theta(x,y)\wedge \forall e\in\dom(\phi(x))\cap\dom(\phi(y)).\\\forall x'\in\phi(x,e).\forall y'\in \phi(y,e).\\ \ell(x')\cap\ell(y')\neq \emptyset\Rightarrow \xi(x')=\xi(y')=\emptyset \vee \xi(x')\cap\xi(y')\neq\emptyset. }} n' \]
%\[GF_\theta(o)=\Set{o\oplus o'|\exists p. \left(\left(o\in\phi(\ngraph,p)\wedge o'\in\phi(\ngraph',p)\right)\vee \left(o\in\phi(\ngraph',p)\wedge o'\in\phi(\ngraph,p)\right)\right)\wedge \theta(o,o')}\]
%
%each object $o$ belongs to all the clustering classes $GC(o)$

%\end{description}

%Let us now discuss some features of the \textbf{grouping} operator: given that this operation relies on the union as it was previously defined (see Definition \vref{def:gsmunion}), it implies that many equivalent representations may be provided more than once. We can observe that, in order to propagate the result of the aggregation, we have to iterate the grouping process until the last leaf is reached. Please also note that the application of such operation to nesting-loop free GSMs allows to express such iterative aggregation with a bounded iteration, by aggregating the already-clustered components as many times as the height of the whole GSM. 



%\hl{In particular, we have to define some operators that allow to both represent the unnesting operator or splicing $\varsigma_S$ for the relational model, and to express an unnesting operator removing all the graph content from one object and placing within the same vertex and edge set.}

%\subsection{Object Operators}
%This set of operations considers each GSM (and hence, each nested graph) as an object, where the generation of a new object derives from the concatenation of the single nested graph object. These operators are in particular defined over Generalized Semistructured Graphs, and hence can be also applied over nested graphs.
%
%In particular, the following class of operations can be considered as formed by a first phase where the reference object resulting from the disjoint union of the two input GSMs ($\texttt{disjoint}(n,n')$) is transformed ($\mu$). Hereby, those two basic operators are required prior to defining the single object transformations.
%
%{\color{red} MISSING}
%
%Now, we want to define an operator that maps each single element into another one, while transforming its content. Consequently, each element contained in the object $\ngraph$ representing $n$ is transformed through functions $f_L$, $f_E$ and $f_C$, respectively associating a new label, expression and nesting content.
%
%{\color{red}MISSING}
%
%At this point, we can express the object concatenation already defined for property graphs' vertices and edges during the definition of graph join.
%
%%operations can be considered as modifying only the object directly provided by the GSM, without modifying the actual information content. As we will se in the following sections, where objects are treated as collections, the creation of new elements requires the creation of new elements within the collection first, to which a new id has to be associated. The first operation is the generalization of the concatenation operator proposed for vertex and edge concatenation within graph joins. This operator can be re-defined over objects as follows:
%
%
%
%
%Given that now objects can even store collections of values, we can generalize the concatenation to a merge operation, allowing to concatenate the id lists from the two objects hold in the same attribute. Consequently, such operation can be defined as follows:
%
%\begin{example}[Merge]
%	Given two GSMs  $n=(\ngraph_c,O,\ell,\xi,\phi)$ and $n'=(\ngraph'_c,O',\ell',\xi',\phi')$, their \textbf{merge}\index{paNGRAm!$\otimes$} $n\otimes n'$ maps their union reference object into a new one where, for each shared collection associated with the same expression $p\in\lang$, the id list shared between the two objects; when $p$ is only present in one of the two GSMs, the id collection is the one coming from the original object. Therefore, the operator is defined as follows:
%\[n\otimes n'=\texttt{map}_{\ell_{nn'},\xi_{nn'},\phi_{nn'}\oplus[\tilde{\ngraph}_c\mapsto \psi_3]}(\texttt{disjoint}(n,n'))\]
%Where $\psi_3$ is defined as follows:
%\[\psi_3(p)=\begin{cases}
%\phi_{nn'}(\ngraph'_c,(1,p))\cup\phi_{nn'}(\ngraph_c,(2,p)) & (1,p),(2,p)\in\dom(\phi_{nn'}(\tilde{\ngraph}_c))\\
%\phi_{nn'}(\ngraph'_c,(2,p)) & (2,p)\in\dom(\phi_{nn'}(\tilde{\ngraph}_c))\\
%\phi_{nn'}(\ngraph_c,(1,p)) & \textup{oth.}\\
%\end{cases}\]
%\end{example}





%\subsection{Object Operations}
%%Given that the map functions that will follow are going to operate on the definition of a single object, it is relevant to define the operations that are possible on Objects. We already introduced the object concatenation in Definition \ref{def:concatenation} as the basic operatior for combining vertices in graph joins. This definition can be promptly be restated for objects as follows:
%%\begin{definition}[Object Concatenation]
%%	Given two objects  $f$ and $g$, their \textbf{concatenation}\index{concatenation!object} $f\oplus g$ is the object returning $f(x)$ if $x\in \texttt{key}(f)$, and $g(x)$ if $x\in \texttt{key}(g)$. $\bot$ is returned otherwise.
%%	Such object concatenation is used when $\forall x\in \texttt{key}(f)\cap \texttt{key}(g). f(x)=g(x)$.
%%\end{definition}
%
%
%
%
%
%We can also define a restriction operator; its definition is similar to the restriction that, within the relational model, each record undergoes if contained inside a relation over which a projection operator is applied.
%
%\begin{definition}[Object Restriction]
%	A \textbf{restriction} \index{restriction!object} of any object $o\in D_O$ over an attribute set $A$ is defined as a function restriction, that makes $\restr{o}{A}$ return only the values that are associated to the keys in both $A$ and in the key set of $o$:
%	\[\restr{o}{A}=k\mapsto\begin{cases}
%	v(k) & k\in A\cap \texttt{key}(o)\\
%	\bot & oth.
%	\end{cases}\]
%
%\end{definition}
%
%We can also define a renaming function for each kind of object as follows:
%
%\begin{definition}[Object Renaming]
%	A \textbf{renaming} \index{renaming!objext} $\rho_{k\to\alpha}(o)$ for an object  $o\in D_O$ generates a new object, where one of its attributes $k\in\texttt{key}(o)$ is renamed as $\alpha\notin\texttt{key}(o)$:
%	\[\rho_{k\to\alpha}(o) = \restr{o}{\texttt{key}(o)\backslash k}\oplus \Braket{\alpha\mapsto o(k)}\]
%\end{definition}
%
%Given that  object in $D_O$ can be seen as a collection of collection of objects, we can even redefine all the operations that have been defined in \cite{magnani04} for object collections:
%
%\begin{definition}[Object Difference]
%Given two objects $f$ and $g$, their difference $f\ominus g$ is the new object containing the key values associations in $f$ that are not represented in $g$:
%\[f\ominus g = k\mapsto\begin{cases}
%	f(k) & k\in\texttt{key}(f)\cap \texttt{key}(g) \wedge f(k)\neq g(k)\\
%	f(k) & k\in\texttt{key}(f)\\
%	\bot & oth.
%\end{cases}\]
%\end{definition}
%
%We can also generalize the collection grouping $\gamma_{SF}$  \cite{magnani04} in order to consider the overlapping non-exclusive clusters, not considered by traditional grouping operations, as follows:
%
%\begin{definition}[Object Multigrouping]
%	Given a function $SF:A\to(\mathcal{L}_{MM}\times \mathcal{P}(\mathbb{N}))\to \mathcal{P}(\mathcal{C})$ associating an object to a set of clusters in $\mathcal{P}(\mathcal{C})$, a \textbf{multigrouping}\index{grouping!object multigrouping} $\gamma_{SF}(o)$ returns a new object where each class $x\in \mathcal{C}$ contains the union of all the key-values that belong to the class $x$, and leaves the other associations unaltered:
%	\[\gamma_{SF}(o)=\left(\restr{o}{\Set{k\in\texttt{key}(o)|SF(k,o(k))=\emptyset}}\right)\quad\oplus\quad \bigoplus_{x\in\mathcal{C}}\left(\bigotimes_{k\in\texttt{key}(o),x\in SF(k,o(k))} \Braket{x\mapsto o(k)}\right)\]
%	Please note that if $SF$ always returns a singleton, the current operation reduces to the original grouping presented in \cite{magnani04}.
%\end{definition}
%
%Finally, we can also define the splitting operator for a single object, where each object identifier is resolved through the $\iota^{-1}$ indexing function. Please note that this is the only function that actually returns a collection of objects instead of a single object: the reason is that this operator will be used inside some other graph operators.
%\begin{definition}[Object Slicing]\label{def:objslicing}
%	Given a restriction set $A$, the \textbf{object slicing}\index{slicing!objects} returns a collection of objects where $\restr{f}{A}$ is extended to each object contained inside the other value fields except the ones in $A$:
%
%	\[\varsigma_{A}(f)=\Set{\restr{f}{A}\oplus g_y|k\in \texttt{key}(f)\backslash A, g_y\in \iota^{-1}(\texttt{ls}(f(k)))}\]
%
%	We can also define the complementary operator, the \textbf{complementary object slicing} by complementing the $A$ set as follows:
%	\[\tilde{\varsigma}_{A}(f)=\varsigma_{\texttt{key}(f)\backslash A}(f)\]
%\end{definition}

