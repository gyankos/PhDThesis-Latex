
%\section{paNGRAm: Use Cases}
\subsection{Representing \textit{is-a} aggregations}\label{subsec:representingisa}

\index{is a!GSM}
Within Section \vref{sec:informationsintegration} and specifically on Examples \ref{ex:8} and \ref{ex:bigraphbibnet}, we have already touched upon structural graph aggregation. We also observed that aggregations on top of graph data structures do not allow drill-down operations on top of aggregated data. Even if the formal operator performing such operation on top of nested graphs will be fully provided in Chapter \vref{cha:nesting}, in the following two subsections we will show how GSM allows to integrate tree (hierarchy) and relational data extracted from a graph, so that it can be drill-down in a second step by the user.

As we  saw in Example \vref{ex:inaggr} where data aggregation operations were briefly introduced, there is a need for representing nested components to which an aggregated representation is provided. In particular, GSMs can explicitly associate to each object $o$ a collection $\phi(o,p)$ of elements over which aggregation functions $p$ can be evaluated over the data in such collection.

\begin{example}[continues=ex:inaggr]
We want to show how \textit{is-a} aggregations are possible by using already existing data hierarchies. Since our data model violates the first normal form - similarly to other semi-structured data models - it is possible to express the hierarchy in Figure \vref{fig:hierarchy} through the following instantiation:
	\[H=(r,\Set{r,t_1,t_2,c_1,\dots,c_4,p_1,\dots,p_8},\ell,\xi,\phi)\]
	The reference object $r$ represents the root of the hierarchy and is defined as follows:
	\[\ell(r)=[\texttt{root}]\quad \phi(r,\texttt{parentof})=[t_1,t_2]\]
	%The indexing function is defined as follows:
	%\[\iota(t_1)=1,\;\iota(t_2)=2,\;\iota(c_1)=3,\;\iota(c_2)=4,\;\dots\;,\iota(p_8)=14\]
%	The labelling function is defined only over the vertices and the $\lambda$ function has an empty domain, because within this model no edges are represented, because all the part-of hierarchy is expressed through the nested components.
%	\[\forall v_i\in V. \ell(v)=\Lambda\qquad \dom(\lambda)=\emptyset\]
	In particular, each element of the hierarchy is defined as follows:
	\[\defzt{t_1}{type}{House cleaner}{parentof}{3,4}\]
	\[\defzt{t_2}{type}{Food}{parentof}{5,6}\]
	\[\defzt{c_1}{category}{Cleaner}{parentof}{7,8}\]
	\[\defzt{c_2}{category}{Soap}{parentof}{9,10}\]
	\[\defzt{c_3}{category}{Diary Product}{parentof}{11,12}\]
	\[\defzt{c_4}{category}{Drink}{parentof}{13,14}\]
	
	\[\defzi{p_1}{product}{Shiny}\qquad \defzi{p_2}{product}{Brighty}\]
	\[\defzi{p_3}{product}{CleanHand}\qquad \defzi{p_4}{product}{Marseille}\]
	\[\defzi{p_5}{product}{Milk}\qquad \defzi{p_6}{product}{Yogurt}\]
	\[\defzi{p_7}{product}{Water}\qquad \defzi{p_8}{product}{Coffee}\]
%\end{example}
%
%
%As we can see, when an attribute contains a constant value, it is mapped as a pair where the expression provides the value and the collection is empty. On the other hand, when we want to remark that our attribute contains a list of identifiers, we use $\cdot$ as an empty expression. The reason of this representation is easy to say: whenever we want to use  this hierarchy as a specific dimension for our data, it will be easy enough to replace the $\cdot$ element with the aggregation function over the nested elements. Please note that this representation is possible neither in the nested relational model, nor within semistructured data.
%
%\begin{example}
%	Let us continue the previous example: 
We now want to use this hierarchy as a dimension for the \texttt{Product} objects in Figure \vref{fig:umlmodelling}. As a start, let us suppose that each \texttt{Product} vertex is associated to a \texttt{quantity} attribute, providing the sum of the \texttt{quantity} of the ingoing \texttt{item} edges. Since such aggregation functions can be only expressed over the data collections and given that there are no particular constraints on the types of the ids to be stored in such data collections, we have that each expression shall provide the following aggregation function ($\xi'[0]$):
\begin{lstlisting}[language=script,basicstyle=\ttfamily\scriptsize]
quantity := (foldl {0,
                    o.phi ["quantity"],
                    x -> { (foldl {0,
                                   map(select(x[0].phi["Attribute"] : y -> {y.ell == "quantity"}))}) } 
                                       : y -> { y.xi[0] } ),
                                   y -> { y[0] + y[1] }
                                  }) }
                   }
            )
\end{lstlisting}
	%\[\texttt{quantity}(x)=\textup{sum}(\phi(x,\texttt{quantity}))\]
In this case we can use $\xi$ to store the definition of \texttt{quantity} function, that will differ from the one used within hierarchies. This other function is going to be defined as follows ($\xi''[0]$):
\begin{lstlisting}[language=script,basicstyle=\ttfamily\scriptsize]
	quantity := (foldl {0, o.phi ["quantity"], y -> {(y[0].xi[0] x[0])+y[1]}})
\end{lstlisting}
In this way, the functions' evaluation may be evaluated directly by traversing the hierarchy, without any required need of effectively aggregating the data.
	As a next step, from the whole company database we can select only the products (filtering through $\sigma$), and then extend each of them with a new \texttt{quantity} attribute aggregating their incoming edges' quantity values. Such statement can be expressed through the following statement:
	\[P'=\texttt{filter}_{x\mapsto \mstr{Product}\in\ell(x)}(\texttt{map}_{\ell,\xi',\phi \oplus_v[[v, [\mstr{quantity}, [v\in \phi(\ngraph_{DB},\RELA)|\exists u.\lambda(e)=(u,v)\wedge \mstr{Product}\in\ell(v)]]]]}(DB))\]
%	where $DB$ can be thought as a nested graph, where the map operator extending the $Calc$ operator in \cite{Calders2006} for arbitrary object mapping is defined as a mapping function over both vertices and edges, and eventually updating the other sets and functions accordingly to the mapping function $f$:
%	\[\textup{map}_f(N)=(\Set{f(c)|c\in V},\Set{f(c)|c\in E},A',\Lambda',\ell',\lambda',\iota')\]
	At this point, we want to do the same operation for the hierarchy: we must first rename \texttt{parentof} into \texttt{quantity}$(x)$, and then just replace the \texttt{parentof} placeholder with the function performing the aggregation:
	\[H'=\texttt{map}_{\ell,\xi'',\phi\oplus_s [[s, [\mstr{quantity},\phi_H(s,\mstr{parentof})]]]}(H)\]
	% \textup{map}_{x\mapsto(\textbf{\texttt{let}} (_,s) = x(\texttt{quantity})\textbf{\texttt{ in }} x(\texttt{quantity}):=(f_{qs},s) )}(\rho_{\texttt{quantity}\leftarrow\texttt{parentof}}(H)) 
	In order to be able to evaluate the aggregation at different hierarchy's abstraction levels, discard all the other informations from the database's products and keep only the given dimension, we can finally perform the following left join (that can be defined from the $\otimes_\theta$ product):
	\[H'\;\leftouterjoin\; \pi_{\texttt{name,quantity}}(P')\]
	Please note that this final join was possible because the products within the hierarchy do not have a \texttt{parentof} field, and hence they have no \texttt{quantity} one. At this point, the amount of the products bought by the company's employees can be done by evaluating the $\texttt{quantity}(x)$ expression over its nested content: the expression's evaluation will recursively visit the remaining part of the hierarchy joined with the data.
\end{example}

