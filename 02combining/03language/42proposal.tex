




%%%%%%%%%%%%%%%%%
%%%%%%%%%%%%%%%%%
%%%%%%%%%%%%%%%%%
\subsection{paNGRAm: Nested Graph Relational Algebra}\label{ssec:ngrahop}
Previously we defined a set of operations to be applied over GSM data representation that have no specific domain constraints. One of the data structures that have such kind of constraints are graphs: an edge cannot generally exist without either a source or a target vertex. Therefore, while defining such operations, we must also make sure that the original model constraints' are preserved. This also implies that some operations, previously defined for general GSM, will be implemented for nested graphs with some restrictions (\texttt{constraint}). The paragraphs contained in this section outline some data operations required (and, in most cases, already defined for such data model).

\phparagraph{Unary operators}
Walking on the footsteps of GSQL, we have to define two different creation operations for both vertices and edges because they have two different requirements, while other value creation operations may directly use the \texttt{create} operation. Concerning the vertices, we have that vertices are specific object contained in a graph's $\ONTA$ collection by definition. It follows that the creation of an object is not a sufficient definition for their creation. Therefore, we must chain the creation operation immediately with a map operation, allowing to immediately insert the object inside a graph (in this case, the reference object).

\begin{definition}[Vertex Creation]
	Given a nested graph $\nested=(\ngraph_c,O,\ell,\xi,\phi)$, the \textbf{vertex creation} operator creates and promotes an object with a fresh id $\omega$ into a vertex by embedding it into a graph $\ngraph'\in\varphi(\ngraph)$. By knowing the object that will contain $\omega$ as a vertex is $\ngraph'$, we can also apply some restrictions to the containts of $f_C$ that is going to define $\phi(\omega)$ by removing from its codomain all the elements containing $\ngraph'$ and $\ngraph'$ itself.
	\[\kappa^{\ngraph'\leftarrow \omega}_{L,F,f_C}(\nested)=\texttt{maps}_{\ell,\xi,\phi\oplus \ngraph'\mapsto\ONTA\mapsto \phi(\ngraph',\ONTA)\cup[\omega]}(\texttt{create}^\omega_{L,F,x\mapsto f_C(x)\backslash[o'\in O|\ngraph'\in \varphi^(o')\vee \ngraph'=o']}(\nested))\]
\end{definition}

We can follow the same approach for creating edges: in this case we can generalize the creation of one single element into a creation of several edges, so that the ``link discovery'' class of operations can be implemented \cite{markus} via the straightforward satisfaction of a $\theta$ predicate among the vertices.

\begin{definition}[Edge Creation (Link Discovery)]
	Given a nested graph $\nested=(\ngraph_c,O,\ell,\xi,\phi)$, the \textbf{edge creation (link discovery)} operator $\dot\kappa^\theta_{L,F,f_C}$ establishes new edge  $d_{u,v}$ s.t. $\lambda(d_{u,v})=(u,v)\in\phi(\ngraph,\ONTA)^2$ having $L$ as a labels set and $F(u,v)$ as a set of expressions, if both vertices vertices $u,v$ satisfy a given predicate $\theta$:
	\[\dot\kappa^{\ngraph'\Leftarrow\theta}_{L,F}(\nested)=\texttt{map}_{\ell_{\nested\nested'},\xi_{\nested\nested'},\phi_{\nested\nested'}\oplus [[\ngraph', \phi(\ngraph')\oplus\,[[\RELA, \phi(\ngraph_c,\RELA)\cup[d_{u_c,v_c}|(u_c,v_c)\in S_\theta]\,]] \;]]}(\texttt{fold}_{S_\theta,f_3}(\nested))\]
	where $S$ is the set containing all the pair object-vertices satisfying the $\theta$ predicate:
	\[S_\theta\eqdef\Set{(u_c,v_c)\in\phi(\ngraph,\ONTA)^2|\theta(u_c,v_c)}\]
	and $f_3$ is the function creating the new edge for each pair in $S_\theta$:
	\[f_3\eqdef((u_c,v_c),\nested')\mapsto\texttt{create}^{d_{u_c,v_c}}_{L,F(u_c,v_c), (x\mapsto f_C(x)\backslash[o'\in O|\ngraph'\in\varphi^*(\ngraph)])\oplus\SRC\mapsto [u_{c+1}]\oplus \DST\mapsto [v_{c+1}]}(\nested')\]
	In particular, for each pair of vertices $d_{u_c,v_c}$ is the function generating the new edge id and is defined as: $(\max O+dt(u,v))_{c}$.  By knowing the object that will contain $\omega$ as a vertex is $\ngraph'$, we can also apply some restrictions to the containts of $f_C$ that is going to define $\phi(\omega)$ by removing from its codomain all the elements containing $\ngraph'$ and $\ngraph'$ itself.
\end{definition}

Please note that the \texttt{create} (Definition \vref{gsql:objcreate}) can be always use to create other non-vertex or non-edge objects. Let us now discuss the \texttt{map} operator for GSMs for transforming vertices and edges: as we previously discussed, the embedding function that can be provided can potentially undermine the nested graph model constraints; therefore, even if \texttt{map} (Definition \vref{gsql:map}) will be still used for defining other nested graph operations (e.g. vertex and edge creation) we may restrict the \texttt{map} operation to the sole label and value update as follows for consistency reasons:

\begin{definition}[(Value and Label) Update]	
	Given an GSM $\fullnested[][\ngraph_{c}]$, the \textbf{(value and label) update}\index{paNGRAm!$\upsilon$} operator $\texttt{map}_{f_L,f_E,f_C}$ associates to each object $o$ represented in $\varphi^*(\ngraph)$, $\ngraph$ included, a new one having labels $f_L(o)$ and expressions $f_E(o)$. Moreover, it associates a new id to all the transformed objects $\delta O$ such that $\delta O=\Set{o\in O|f_L(o)\neq \ell(o)\vee f_E(o)\neq \xi(o)}$
	\[\upsilon_{f_L,f_E}(\nested)=\texttt{map}_{f_L,f_E,\phi}(\nested)\]
\end{definition} 

This decision also implies that filtering $\sigma$ operations are not directly possible within this model. As we previously observed, such operations will be generalized through (nested graph) traversal operations as observed in the previous section for NautiLOD, through which edges may also be removed using a GSM traversing semantics. %As we will see later on, the vertex and edge containment will be provided by other more structured operators. 


Continuing in the same steps of GSQL, we can \texttt{elect} the graph to be used as a reference object, and hence we can select over which we can work on. Please note that given that the GSM nesting-loop freeness condition applies to any object of the GSM, the \texttt{elect} operation does not undermines the model constraints for the GSM. Therefore, the very same operator in Definition \vref{gsql:elect} may be used. 


Similar considerations may be also applied to the \texttt{fold}  (Definition \vref{def:fold}) and $Calc$ (Definition \vref{def:calc}). The generic graph nesting operation is going to be defined in Chapter \vref{cha:nesting}, while the unnesting operator is going to be defined in the following paragraph.

\paragraph*{$n$-ary operators}
As previously observed, the class of graph $\otimes_\vartheta$ product may be defined as a vertex join operation alongside with the creation of new edges among the returned and matched vertices by using a specific semantics, \textbf{es}. In the context of the nested graphs' edge creations, the semantics is represented by a $\Braket{\textbf{es},L,F}$ triple required by the $\dot\kappa^{\ngraph'\Leftarrow\textbf{es}}_{L,F}$ operator. Therefore, the graph join operation can be defined as follows:

\begin{definition}[(Nested) Graph $\otimes_\vartheta$ Product]
	Given two nested graph operands, $\nested$ and $\nested'$, a $\vartheta$ predicate over the edges and an edge semantics $\Braket{\textbf{es},L,F}$, such operator can be represented as follows:
	\[\nested\otimes_\vartheta^{\Braket{\textbf{es},L,F}}\nested'=\dot\kappa^{\ngraph'\Leftarrow\textbf{es}}_{L,F}(\nested\otimes_{(x,y)\mapsto x,y\in\phi(\ngraph',\ONTA)\wedge\vartheta(x,y)}\nested')\]
\end{definition}


On the other hand, the \texttt{disjoint} union (Definition \vref{gsql:disjoint}) clearly breaks the model constraints for nested graphs, because it doesn't preserve the containment attributes in which vertices and edges are stored. Therefore, we can use such operator only in combination with other operators that will change the structure by subsequently using a \texttt{map} operator. Therefore, instead of directly implementing it, we can continue our discussion with the set operators outlined in Subsection \vref{ssec:gsmsop}. We already observed that such operations play a double role of both object and collection operators, due to the object collection-attributes' values dualism. Nevertheless, the intersection and difference operations may also require that the edges must be always checked if may actually link some nodes appearing in a $\phi(o,\ONTA)$ within the final result. Therefore, we must define this intermediate operator assuring the consistency of the set operations:

\begin{definition}[Nested Graph Constraint]
The \texttt{constraint} operation over a nested graph $\nested$ assures that all the edges appears as nodes within one of the objects of $\varphi^*(\ngraph)$. This operator can be defined via a \texttt{map} operator as follows:
\[\texttt{constraint}(\nested)=\texttt{map}_{\ell,\xi,\phi\oplus [[\ngraph,[\genfrac{}{}{0pt}{}{[\ONTA,\phi(\ngraph,\ONTA)],}{[\RELA,[e\in\phi(\ngraph,\RELA)|\exists o,o'\in \varphi(\ngraph).\lambda(e)\in\phi(o,\ONTA)\times\phi(o',\ONTA) ]]}]]]}(\nested)\]
\end{definition}

Please note that a combination of \texttt{constraint} with a \texttt{fold} recursion may also ensure that each graph within the nested graph $\nested$ satisfies the model's requirements. Moreover, observe the similarity between the former operator and the edge restrictions for graph grammars as in Defnition \vref{def:graphgrammar}.

Now we can define the set operations over graphs by combining the set operations with the constraint operators as follows:
\[\bigsqcup^\omega_{1\leq i\leq n}\nested_i=\texttt{constraint}\left(\bigcup^\omega_{1\leq i\leq n}\nested_i\right)\quad \nested\backslash\backslash^{\omega}\nested' = \texttt{constraint}(\nested\backslash^{\omega}\nested')\quad \bigsqcap^\omega_{1\leq i\leq n}\nested_i=\texttt{constraint}\left(\bigcap^\omega_{1\leq i\leq n}\nested_i\right)\]
The \texttt{constraint} operator may also be adopted within the (unary) unnesting definition for nested graphs: in this case, we want only to unnest the vertices and the edges' contents appearing on the graphs represented as the object reference. Such raw operator may be defined as follows:

\[{\mu_2}_{P}^{\textbf{keep}}(\nested)=\texttt{map}_{\ell,\xi,\phi\oplus \left[\left[\ngraph,\left[\genfrac{}{}{0pt}{}{[\ONTA,[o'\in\phi(\ngraph,\ONTA)|\neg P(o')\wedge \textbf{keep}]\cup\bigcup_{o\in\phi(\ngraph,\ONTA),P(o)}\phi(o,\ONTA)],}{[\RELA,[o'\in\phi(\ngraph,\RELA)|\neg P(o')\wedge \textbf{keep}]\cup\bigcup_{o\in\phi(\ngraph,\RELA),P(o)}\phi(o,\RELA)]}\right]\right]\right]}(\nested)\]



%We already observed with Example \vref{ex:partof} that nested graphs also provide the graph-collection graph dualism. Therefore, a desired property for our $n$-ary graph operators should be show the same kind of dualism.
%
%At page \pageref{def:otimesthetaList} we already discussed an operator which provided an arbitrary combination over $n\times m$ sets of elements, both through selection ($\theta$) and by differently combining the matched elements ($\otimes$). The need of these kind of operators is confirmed by 
%
%{\color{red} [TODO]
%	\begin{itemize}
%		\item generica operazione di composizione di due insiemi di grafi, come in NautiLOD. osservazione: gli aggiornamenti degli archi sono arbitrari, e si effettuano successivamente.
%		\item Osservazione, che unione e join possono essere definite similarmente.
%		\item Definizione di set operations
%		\item Definition di join operations
%\end{itemize}}
%
%After defining the semistructured embedding operation, we must ask ourselves what does it mean to embed a graph into another object, either a vertex or an edge. Similarly to the semistructured operation, we want to extend each element with already-existing elements within the database. In order to do so, we can define it as the combination of the pattern matching operator alongside with the expression $\varepsilon_{\smatch(m_P(\nested),t)}(P)$, associating the result of the matching on $\nested$ over the (nested) graph query $P$ over $P$ itself through the $\smatch$ function previously defined in Definition \vref{sec:graphamatch}. At this point we can even draw a parallel between the $Pop$ operator defined for the data mining algebra (Section \vref{subsec:dmalgebra}) and the previously instantiated embedding operator.
%Given that this operator performs an associations between intension $P$ as a graph and data $\nested$, we have that this operator acts as the $Pop$ operator within the data mining (relational) algebra. Hereby, we can use the following shorthand when we want to made explicit that the graph embedding performs an association between an intensional property, expressed as a graph, and other graph data:
%\[Pop_m(\nested,P)=\varepsilon_{\smatch(m_P(\nested),t)}(P)\]
%
%As we can see, the $\kappa$ operator allows to create just one single element. In order to create more than one object, we must chain multiple $\kappa$ at a time. One way to do so is to list all the objects that allow the creation of the new element, and then $fold$ the input nested graph with such $\kappa$ operator. We can use the same approach for the definition of the link discovery operator, defined as follows:
%
%
%
%Graph can be thought as link discovery processes where, instead of creating edges between the matching elements, new vertices are produced.
%\[(\nested,\nested')_V\eqdef\texttt{map}_{\ell_{\nested\nested'},\xi_{\nested\nested'},\phi_{\nested\nested'}\oplus[\omega_c\mapsto\ONTA\mapsto [d_{u_c,v_c}|(u_c,v_c)\in S_\theta]]}(\kappa^\omega_{\ell(\ngraph),\xi(\ngraph),[]}(\textup{fold}_{J_\theta,f_4}(\nested\cup\nested')))\]
%where $J_\theta$ is the set returning all the vertices from both graphs to be joined:
%\[\begin{split}
%J_\theta\eqdef\{(u_c,v_c)\in\phi(\ngraph,\ONTA)\times\phi(\ngraph',\ONTA)|&\theta(u_c,v_c)\wedge \\
%&\forall p\in \dom(\phi(u_c))\cap \dom(\phi(v_c)).\phi(u_c,p)=\phi(v_c,p)\}\\
%\end{split}\]
%while $f_4$ is the function actually generating the new combined vertex through the $\phi$ function composition:
%\[f_4\eqdef((u_c,v_c),\nested')\mapsto\kappa^{d_{u_c,v_c}}_{L,F(u_c,v_c),\phi(u_c)\oplus\phi(v_c)}(\nested')\]
%Then, in a subsequent step, edges can be created according to an edge semantics \textbf{es} expressed through a link discovery operator $\dot{\kappa}^{\textbf{es}}_{L_E,F_E}$. Please note that, given that the combined vertices as a result of the joining phase preserve the original vertices id, the original vertices can be always accessed by the graph. Therefore, the graph join operator can be defined as follows:
%
%\[\nested\Join_\theta^{\textbf{es}}\nested'=\dot\kappa^{\textbf{es}}_{L_E,F_E}((\nested,\nested')_V)\]
%
%Note that, when the nested graph contains no edges, such operation reduces to the relational join over first-level entities.
%
%
%Given that such operator allows to perform aggregations only over specific parts of the nested graph, such GSM operator can be even used to implement the nested graph multigroup by, thus allowing to generalize the grouping operator proposed in \cite{JunghannsPR17} \hl{\texttt{[TODO] Please note that the $\varsigma_S$ operator cannot express the graph unnesting for $S\neq \emptyset$, because the $\varsigma_S$ only joins the nested components to the remaining part. Moreover, we still need to define an explicit function performing the data extraction from the nested components, while leaving to the user the final destination of such extracted data.}}.

\phparagraph{Graph Data Mining operators}
With reference of the three world data mining model, we may observe that graph representations allow to collapse the three distinct data worlds into one. In Chapter \vref{cha:join} we observed that graphs may represent both \textsc{Data}\index{D} and \textsc{Model}s \index{M}, where -- as observed in the succeeding chapters -- the latter may be also generalized and interpreted as \textsc{MetaModel}s \index{MM} because such models may be used as query languages. With back reference to the relational Data Mining Algebra, the $\kappa$ data mining operator allowing to extract an intensional representation from the data one may be interpreted through the $\alpha_1$ operator allowing the extraction of a graph schema from the given graph data. The mining loop $\lambda$ operator may be defined via the higher order \texttt{fold} operator, which allows to perform iterations over arbitrary data collections. Last, the $Pop$ operator allowing to associate each data representation to the schema it has to match may be represented as follows: we may extend the intensional representation of a graph schema (or pattern) $P$ with the data it matches by extending each object $o$ in $P$ with the objects $f_j(o)$ in $\nested$ resulting from the morphism $f_j$ generated by the matching process $m_P(\nested)$.
\begin{equation}
Pop_m(\nested,P)=\varepsilon_{o\mapsto\bigoplus_{f_j\in m_P(\nested)}[[\mstr{j},f_j(o) ]]}(P)\label{eq:popGraph}
\end{equation}
Therefore, extensional information provided by the $Pop$ operator is represented by nested graphs: if we unnest the extensional information we may obtain the original data values with its original schema ($\pi_A\eqdef{\mu_2}^{\texttt{\color{NavyBlue}ff}}_{\texttt{\color{NavyBlue}tt}}$) \index{$\pi_{A}$!in GSQL}
while the $\pi_{RDA}$\index{$\pi_{RDA}$!in GSQL} operation which forgets the nested information may be defined by a simple map operation which forgets all the morphism-nested information within the nested graph and retains the objects that have been actually matched in the previous phase, thus acting  as projection operator. Therefore, the GSQL algebra even allows to represent the data mining operations provided in the previous formal model, alongside with the standard relational and set operations.

%\section{paNGRAm: Use Cases}
\subsection{Representing \textit{is-a} aggregations}\label{subsec:representingisa}

\index{is a!GSM}
Within Section \vref{sec:informationsintegration} and specifically on Examples \ref{ex:8} and \ref{ex:bigraphbibnet}, we have already touched upon graph aggregation operations. We also observed that aggregations on top of graph data structures do not allow drill-down operations on top of aggregated data. Even if the formal operator performing such operation on top of nested graphs will be fully provided in Chapter \vref{cha:nesting}, in the following two subsections we will show how the proposed data structure allows to nest data, so that it can be drill-down in a second step by the user that wants to explore the data model, without using any unnseting operator but by simply traversing the GSM data structure.

As we already saw in Example \vref{ex:inaggr} where data aggregation operations were briefly introduced, there is a need for representing nested components to which an aggregated representation is provided. In particular, GSMs can explicitly associate to each object $o$ a collection $\phi(o,p)$ of elements over which aggregation functions $p$ can be evaluated over the data in such collection.

\begin{example}[continues=ex:inaggr]
We want to show how \textit{is-a} aggregations are possible by using already existing data hierarchies. Since our data model violates the first normal form - similarly to other semi-structured data models - it is possible to express the hierarchy in Figure \vref{fig:hierarchy} through the following instantiation:
	\[H=(r,\Set{r,t_1,t_2,c_1,\dots,c_4,p_1,\dots,p_8},\ell,\xi,\phi)\]
	The reference object $r$ represents the root of the hierarchy and is defined as follows:
	\[\ell(r)=[\texttt{root}]\quad \phi(r,\texttt{parentof})=[t_1,t_2]\]
	%The indexing function is defined as follows:
	%\[\iota(t_1)=1,\;\iota(t_2)=2,\;\iota(c_1)=3,\;\iota(c_2)=4,\;\dots\;,\iota(p_8)=14\]
%	The labelling function is defined only over the vertices and the $\lambda$ function has an empty domain, because within this model no edges are represented, because all the part-of hierarchy is expressed through the nested components.
%	\[\forall v_i\in V. \ell(v)=\Lambda\qquad \dom(\lambda)=\emptyset\]
	In particular, each element of the hierarchy is defined as follows:
	\[\defzt{t_1}{type}{House cleaner}{parentof}{3,4}\]
	\[\defzt{t_2}{type}{Food}{parentof}{5,6}\]
	\[\defzt{c_1}{category}{Cleaner}{parentof}{7,8}\]
	\[\defzt{c_2}{category}{Soap}{parentof}{9,10}\]
	\[\defzt{c_3}{category}{Diary Product}{parentof}{11,12}\]
	\[\defzt{c_4}{category}{Drink}{parentof}{13,14}\]
	
	\[\defzi{p_1}{product}{Shiny}\qquad \defzi{p_2}{product}{Brighty}\]
	\[\defzi{p_3}{product}{CleanHand}\qquad \defzi{p_4}{product}{Marseille}\]
	\[\defzi{p_5}{product}{Milk}\qquad \defzi{p_6}{product}{Yogurt}\]
	\[\defzi{p_7}{product}{Water}\qquad \defzi{p_8}{product}{Coffee}\]
%\end{example}
%
%
%As we can see, when an attribute contains a constant value, it is mapped as a pair where the expression provides the value and the collection is empty. On the other hand, when we want to remark that our attribute contains a list of identifiers, we use $\cdot$ as an empty expression. The reason of this representation is easy to say: whenever we want to use  this hierarchy as a specific dimension for our data, it will be easy enough to replace the $\cdot$ element with the aggregation function over the nested elements. Please note that this representation is possible neither in the nested relational model, nor within semistructured data.
%
%\begin{example}
%	Let us continue the previous example: 
We now want to use this hierarchy as a dimension for the \texttt{Product} objects in Figure \vref{fig:umlmodelling}. As a start, let us suppose that each \texttt{Product} vertex is associated to a \texttt{quantity} attribute, providing the sum of the \texttt{quantity} of the ingoing \texttt{item} edges. Since such aggregation functions can be only expressed over the data collections and given that there are no particular constraints on the types of the ids to be stored in such data collections, we have that each expression shall provide the following aggregation function ($\xi'[0]$):
\begin{lstlisting}[language=script,basicstyle=\ttfamily\scriptsize]
quantity := (foldl {0,
                    o.phi ["quantity"],
                    x -> { (foldl {0,
                                   map(select(x[0].phi["Attribute"] : y -> {y.ell == "quantity"}))}) } 
                                       : y -> { y.xi[0] } ),
                                   y -> { y[0] + y[1] }
                                  }) }
                   }
            )
\end{lstlisting}
	%\[\texttt{quantity}(x)=\textup{sum}(\phi(x,\texttt{quantity}))\]
In this case we can use $\xi$ to store the definition of \texttt{quantity} function, that will differ from the one used within hierarchies. This other function is going to be defined as follows ($\xi''[0]$):
\begin{lstlisting}[language=script,basicstyle=\ttfamily\scriptsize]
	quantity := (foldl {0, o.phi ["quantity"], y -> {(y[0].xi[0] x[0])+y[1]}})
\end{lstlisting}
In this way, the functions' evaluation may be evaluated directly by traversing the hierarchy, without any required need of effectively aggregating the data.
	As a next step, from the whole company database we can select only the products (filtering through $\sigma$), and then extend each of them with a new \texttt{quantity} attribute aggregating their incoming edges' quantity values. Such statement can be expressed through the following statement:
	\[P'=\texttt{filter}_{x\mapsto \mstr{Product}\in\ell(x)}(\texttt{select}_{\ell,\xi',\phi \oplus_v[[v, [\mstr{quantity}, [v\in \phi(\ngraph_{DB},\RELA)|\exists u.\lambda(e)=(u,v)\wedge \mstr{Product}\in\ell(v)]]]]}(DB))\]
%	where $DB$ can be thought as a nested graph, where the map operator extending the $Calc$ operator in \cite{Calders2006} for arbitrary object mapping is defined as a mapping function over both vertices and edges, and eventually updating the other sets and functions accordingly to the mapping function $f$:
%	\[\textup{map}_f(N)=(\Set{f(c)|c\in V},\Set{f(c)|c\in E},A',\Lambda',\ell',\lambda',\iota')\]
	At this point, we want to do the same operation for the hierarchy: we must first rename \texttt{parentof} into \texttt{quantity}$(x)$, and then just replace the \texttt{parentof} placeholder with the function performing the aggregation:
	\[H'=\texttt{map}_{\ell,\xi'',\phi\oplus_s [[s, [\mstr{quantity},\phi_H(s,\mstr{parentof})]]]}(H)\]
	% \textup{map}_{x\mapsto(\textbf{\texttt{let}} (_,s) = x(\texttt{quantity})\textbf{\texttt{ in }} x(\texttt{quantity}):=(f_{qs},s) )}(\rho_{\texttt{quantity}\leftarrow\texttt{parentof}}(H)) 
	In order to be able to evaluate the aggregation at different hierarchy's abstraction levels, discard all the other informations from the database's products and keep only the given dimension, we can finally perform the following left join (that can be defined from the $\otimes_\theta$ product):
	\[H'\;\leftouterjoin\; \pi_{\texttt{name,quantity}}(P')\]
	Please note that this final join was possible because the products within the hierarchy do not have a \texttt{parentof} field, and hence they have no \texttt{quantity} one. At this point, the amount of the products bought by the company's employees can be done by evaluating the $\texttt{quantity}(x)$ expression over its nested content: the expression's evaluation will recursively visit the remaining part of the hierarchy joined with the data.
\end{example}

