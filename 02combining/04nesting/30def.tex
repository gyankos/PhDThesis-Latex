\section{Graph Nesting}\label{sec:nestingdef}
As we saw in the previous chapter (Section \vref{ssec:ngrahop}), the (nested) graph operators may be defined as  combinations of the semistructured (and relational) operatorsr. The most general definition generalizes the semistructured nesting approach for arbitrairly aggregating vertices and edges together in similar clusters, that then are going to be recombined:

\begin{definition}[Graph Nesting]
	\label{def:graphnesting}
For every nested graph $\nested$, given a clustering function $CL$ returning set of pairs $k=(a,b)$, where $b$ is the nested element identifier, either vertex ($a=\texttt{e}$) or edge ($a=\texttt{r}$) in which the element is going to be nested to and two (pairs of) transcoding functions $UDF_V=\Braket{f_E^V,f_C^V}$ and $UDF_E=\Braket{f_E^E,f_C^E}$ for transforming the clusters into objects representing vertices and edges, the graph nesting operator $\nu_g$ is defined by the following procedural steps: {(\textbf{i})} we must first group each vertices and edges belonging to the same cluster $(a,b)$ with the same label through $CL$, and within each cluster distinguish the  objects represent ingeither entities or relationships as marked in the following $\alpha_2$'s containment function:
\[\nu_1^{CL,\textbf{keep}} = {\alpha_2^{\textbf{keep}}}_{CL,(k,s)\mapsto [k],(k,s)\mapsto [k],(k,s)\mapsto \left[\tsub{\textup{\textbf{if } }s\subseteq \phi(o.\nested,\ONTA)\textup{ \textbf{then} }\\ \qquad [\ONTA, s]\\ \textup{\textbf{else if } }s\subseteq \phi(o.\nested,\RELA)\textup{ \textbf{then} }\\ \qquad [\RELA, s]\\ \textup{\textbf{else } } []}}\right]}(\nested)\]
(\textbf{ii}) then, we merge them together within the same collections, and merge them into objects by matched cluster, thus keeping entities and relationships separated:
\[\nu_2=\gamma^{\color{NavyBlue}\texttt{tt}}_{\widetilde{GF}}(\texttt{elect}_\omega(\texttt{create}_{[],[],[[\mstr{elements},\phi(\ngraph,\ONTA)\cup\phi(\ngraph,\RELA)]]}^\omega(\nu_1^{CL,\textbf{keep}})))\]
In particular, we have that $\widetilde{GF}$ performs this further aggregation of the nested objects by the cluster labels returned by $CL$ as follows:
\[\widetilde{GF}(o)=\begin{cases}
\ell(x) & \exists b. \ell(o)\subseteq \cod(CL)\\
\emptyset & \textup{oth.}\\
\end{cases}\]
(\textbf{iii}) Then, we transform the object representing clusters either to entities ($V$) or to relationships ($E$) dependingly on the first component of the object's label. Therefore, we shall use the transformation functions:
\[\nu_3=\texttt{map}_{\ell,o\mapsto\begin{cases}
	 f_E^V(o) & \exists b. \ell(o)=[(\texttt{e},b)]\\
	 f_E^E(o) & \exists b. \ell(o)=[(\texttt{r},b)]\\
	 \xi(o) & \textup{oth.}\\
	\end{cases},o\mapsto\begin{cases}
	f_C^V(o) & \exists b. \ell(o)=[(\texttt{e},b)]\\
	f_C^E(o) & \exists b. \ell(o)=[(\texttt{r},b)]\\
	\phi(o) & \textup{oth.}\\
	\end{cases}}(\nu_2)\]
(\textbf{iv}) Last, we create a new element where the clustered objects are separated by the first component of their label, representing if they are either entities or relationships as follows:
\[{\nu_g\;}^{\textbf{keep},\omega}_{CL,UDF_V,UDF_E}(\nested)=\texttt{create}_{[],[],\left[\substack{\ONTA\mapsto \Set{x\in\varphi(o.\nu_2)|x\in \phi(o.\nu_1^{CL,\textbf{keep}},\ONTA)\vee\exists b. \ell(x)=(\texttt{e},b)}\\ \RELA\mapsto \Set{x\in\varphi(o.\nu_2)|x\in \phi(o.\nu_1^{CL,\textbf{keep}},\RELA)\vee\exists b. \ell(x)=(\texttt{r},b)}}\right]}^\omega(\nu_3)\]

%In particular, $CL$ associates to each element of $o.\nested$ a set of pairs $(a,b)$; a pair $(\texttt{e},b)$ remarks that the element is going to be part of a nested vertex $b$, while $(\texttt{r},b)$ remarks that the element is going to be associated to a nested edge $b$.
\end{definition}

If we want to nest a graph by using the matched patterns as described in the introduction, we must express $CL$ as a graph pattern matching classifier, checking whether the given vertex or edge would match the pattern, and returning the cluster identifiers to which it belongs. Hereby, the $CL$ definition must contain the definition for both $g_V$ and $g_E$, and then check whether each node will belong to one single generated cluster. Please also note that we can perform  pattern matching techniques over a single nested graph $\nested$ as a consequence of the implementation of such languages in GSQL, as described in \vref{traversalDef}. Consequently, in this case we can optimize the aforementioned general approach to nesting graphs (that can be also used for graph grouping) into a more efficient operator, which only requires to generate two distinct sets of graph collections that, then, will be used to create the new nested graphs. %Consequently, the following definition can be also given:

%\[\texttt{map}_{\ell,\xi,\phi\oplus [o.\nested_{c+1}\mapsto [\ONTA\mapsto \varphi(o.\nested_{c+1})]]}(\sigma_{s,\gamma_V}(\nested))\cup (\texttt{map}_{\ell,\xi,\phi\oplus [o.\nested_{c+1}\mapsto [\RELA\mapsto \varphi(o.\nested_{c+1})]]}(\sigma_{s',\gamma_E}(\nested)))\]

In particular, we can freely assume that our nested graph pattern matching semantics $s'$ acts as an UDF function, and hence associates to each graph cluster matched by $g_E$ a source and a target vertex. On the other hand, while the previous formal definition of the graph nesting operator provides a general definition matching with Algorithm \vref{alg:generalNesting}, the following  algorithms allow to match the class of graph nesting optimizable problems that is going to be defined in the next section.