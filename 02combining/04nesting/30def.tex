\section{Nested Graphs}
\label{sec:model}
%The term \textit{property graph}  \cite{angles12} usually refers to a directed, labelled and attributed multigraph. 
% In other words, if there is a schema, each vertex and edge will be represented by a relational tuple and without, by a document of key-value pairs. 
%In a property graph a collection of \textit{labels}  is associated to every vertex and edge (e.g., \texttt{[\mstr{Author}]} or \texttt{[\mstr{coAuthorsip}]}). Further on, vertices and edges may have arbitrary named attributes (\textit{properties}) in the form of key-value pairs (e.g., \texttt{\textbf{name}:Baldwin} or \texttt{\textbf{surname}:Oliver}). Property-value associations of vertices and edges can be represented by relational tuples; this is a common approach in literature used even when graphs have no fixed schema \cite{angles12}.
In this chapter we try to define the nested graph data model independently from the GSM and GSQL query language. This choice will result in a limitation within the definition of the data structure and on the operators' formalization. We now define the\textit{ nested (property) graph database} from scratch as the following extension of the property graph data model for nested information:

\begin{definition}[Nested Graph DataBase]
	Given a set $\Sigma^*$ of strings,
	a \textbf{nested (property) graph database} $G$ is a tuple $G=\Braket{\VS, \ES, \lambda,\ell,\omega,\nestF,\prov}$, where $\VS$ and $\ES$ are disjoint sets, respectively referring to vertex and edge identifiers $o\equiv i_c\in\mathbb{N}$; $c$ is an incremental unique number associated to each graph as in the GSM model. 
	
	A function $\lambda\colon \ES\to \VS^2$ maps each edge to its source and target vertex. Each vertex and edge is assigned to multiple possible labels through the labelling function $\ell:\VS\cup \ES\to \wp(\Sigma^*)$.  $\omega$ is a function mapping each vertex and edge into a relational tuple.
	
	In addition to the previous components defining a property graph, we also introduce functions representing \textit{vertex members} $\nestF\colon (\VS\cup \ES)\to\wp(\VS)$ and \textit{edge members} $\prov\colon(\VS\cup \ES)\to \wp(\ES)$. These functions induce the nesting by associating a set of vertices or edges to each vertex and edge. Each vertex or edge $o\in V\cup E$ induces a \textbf{nested (property) graph} as the following pair:
	\[G_o=\Braket{\nu(o),\Set{e\in\epsilon(o)|\lambda(e)\in (\cup_{n\geq 0}\;{\nu\epsilon}^{(n)}(\{o\}))^2}}\]
	where ${\nu\epsilon}$ returns the vertices contained in both vertices and edges ($\nu\epsilon(x)=\nu(x)\cup \nu(\epsilon(x))$). We denote $f(X){:=}\bigcup_{x\in X} f(x)$ when $X\subseteq \textup{dom}(f)$
\end{definition}


As we previously observer, nested graphs can be also implemented in the GSM model. Since the member functions $\nu$ and $\epsilon$ induce the expansion of each single vertex or edge to a graph, we must avoid recursive nesting to support expanding operations.
% This condition is fundamental in order to define different levels as different abstractions over the data. 
Therefore, we additionally introduce the following constraints to be set at a nested property graph database level:

\begin{axiom}[Recursion Constraints]
	For each correctly nested property graph, each vertex $v\in \VS$ must not contain $v$ at any level of containment of $\nu$ and, any of its descendants $m$ must not contain $v$:
	\[\forall v\in \VS. \forall m\in \nu^+(v).\;\; m\neq v\wedge v\notin \cup_{n\geq 1}\;{\nu\epsilon}^{(n)}(m)\]
	Similarly to vertices, any edge shall not contain itself at any nesting level:
	\[\forall e\in \ES. \forall m\in \epsilon^+(e). m\neq e\wedge e\notin \cup_{n\geq 1}\;{\epsilon\nu}^{(n)}(m)\]
	where ${\epsilon\nu}$ returns the edges contained in both vertices and edges ($\epsilon\nu(x)=\epsilon(x)\cup \epsilon(\nu(x))$)
\end{axiom}

Please also note that this model has more restrictive constraints than the ones in the GSM model. This is due to the fact that GSM nested graphs differentiates vertices and edges by containing axioms while, in this case, we must restrict the edges to only the ones that are contained within the strongly nested components of the single nested graph. Nonetheless, a vertex $v$ having a non-empty vertex or edge members is called \textbf{nested vertex}, while vertices with no members are simply referred to \textbf{simple vertices}. For edges, we respectively use the terms \textbf{nested edges} and \textbf{simple edges}. 

\begin{example}[label=exImpl]
	The property graph in Figure \ref{fig:inputbibex2} can be represented by the graph $G_{({{11}_0})}$, which is a nested vertex contained in the following nested graph database:
	\[G=\Braket{\{0_0,1_0,\dots,5_0,11_0\}, \{6_0,\dots, 10_0\},\lambda,\ell,\omega,\nu,\epsilon}\]
	
	The nested vertex $({{11}_0})$  represents a \mstr{Bibliography} graph ($\ell({{11}_0})=[\mstr{Bibliography}]$), to which an empty tuple is associated ($\omega({{11}_0})=\{\}$). Its vertex ($\nu$) and edge ($\epsilon$) members are defined as follows:
	\[\nu({{11}_0})=\{{{0}_0},\dots,{{5}_0}\}\quad\epsilon({{11}_0})=\{{{6}_0},\dots,{{10}_0}\}\]
	
	The simple edge $6$ within the property graph in Figure \ref{fig:inputbibex2} ($\nu({{6}_0})=\epsilon({{6}_0})=\emptyset$) has now id $({{6}_0})$; it has one label, $\ell({{6}_0})=[\mstr{AuthorOf}]$, and it is associated to an empty tuple ($\omega({{6}_0})=\{\}$).
	The source and target vertices are 
	$\lambda({{6}_0})=\Braket{{{0}_0},\;3_0}$. Similar considerations can be carried out for each  remaining edge.
	
	The simple vertex $0$ in the same Figure has id $({{0}_0})$ in the present example; such vertex refers to the \mstr{Author} \texttt{Abigail Conner}. This information is represented as follows:
	\[\ell({{0}_0})=[\mstr{Author}]\quad\nu({{0}_0})=\epsilon({{0}_0})=\emptyset\] \[\omega({{0}_0})=\{\texttt{\textbf{name}}\colon\texttt{Abigail},\texttt{\textbf{surname}}\colon\texttt{Conner}\}\]
	Similar considerations can be carried out for each remaining vertex.
	
\end{example}


\section{Graph Nesting}\label{sec:nestingdef}
The graph nesting operator uses a classifier function grouping all the vertices and edges that shall appear as a member of a cluster $C$. 

\begin{definition}[Nested Graph Classifier, $g_\kappa$]
	Given a set of cluster labels $\;\mathcal{C}$, a \textbf{nested graph classifier} function $g_\kappa$ maps a nested graph $G_o$ into a nested  graph collection $\{G_C\}_{C\in\mathcal{C},G_C\neq \emptyset}$ of subgraphs of $G_o$. Such function uses a classifier function $\kappa\colon \VS\cup \ES\to \wp(\mathcal{C})$ mapping each vertex or edge in either no graph or at least one non-empty subgraph. Each nested graph $G_C$ is a pair
	$G_C=\Braket{\VS_C,\ES_C}$
	where $\VS_C$ (and $\ES_C$) is the set of all the vertices $v$ (and edges $e$) in $G_o$ having $C\in \kappa(v)$ (and $C\in \kappa(e)$). Therefore, the nested graph classifier is defined as follows:
	\[g_\kappa(G_o)=\Set{\Braket{\VS_C,\ES_C}|C\in\mathcal{C},(\VS_C\neq\emptyset\vee\ES_C=\emptyset)}\]
\end{definition}

The former definition is also going to express graph pattern evaluations, where $\kappa$ may be represented as a graph (cf. Neo4J). This assumption allows us to use the graphs in Figure \ref{fig:patterns} as possible $\kappa$. When $\kappa$ is a graph, we denote as $\kappa\xrightarrow{f_C} G_C$ the function $f_C$ associating  each vertex (and edge) in $\kappa$ to possibly more than one vertex (and one edge) in a subgraph $G_C\in g_\kappa(G_o)$. In order to represent the latter subgraphs  as either vertices and edges, we may use
the following \textsc{User-Defined Functions}:
\begin{definition}[User-Defined Functions]
	An \textbf{object user defined function} $\mu_\Omega$ maps each subgraph $G_C\in g_\kappa(G_o)$ into a pair $\mu_\Omega(G_C)=(L,t)$, where $L\in\wp(\Sigma^*)$ is a set of labels and $t$ is a relational tuple.
	
	An \textbf{edge user defined function} $\mu_E$ maps each subgraph $G_C\in g_\kappa(G_o)$ into a pair of identifiers $\mu_E(G_C)=(s,t)$ where $s,t\in\mathbb{N}$.
\end{definition}

\begin{example}
	Within our use case scenario, $\mu_\Omega$  must associate  the authors' informations to each nested vertex resulting from $g_{V}(G_{o})$ , and create nested edges with \textsc{coAuthorship} label and no associated tuple:
	\[\mu_\Omega(G_C)=\begin{cases}
	([\mstr{coAuthorship}],\;\emptyset) & G_C \in g_E(G_{o})\\
	(\ell(f_C(\gamma_V))),\;\omega(f_C(\gamma_V))) & G_C \in g_V(G_{o})\\
	\end{cases}\]
\end{example}

While $\mu_\Omega$ may be used for transforming subgraphs to both vertices and edges, $\mu_E$ is only used to map subgraphs to edges.  
In order to complete such transformation, we have to map each graph in $g_\kappa(G)$ into a new id $\textbf{i}_{\textbf{c}}\notin \VS\cup \ES$, for which an indexing function $\iota_G$ over each $G_C$ has to be defined within our specific task. As we will see in the next section, our scenario provides some constraints on both patterns; this allows the definition of an indexing function  uniquely associating each matched subgraph  $G_C$  to the grouping references' ids.
The previous functions are involved in the definition of our general graph nesting operator:



\begin{definition}[Graph Nesting]
	Given a nested graph $G_{i_c}$ within a nested graph database $G$, an object user defined function $\mu_\Omega$, an edge user defined function $\mu_E$ and an indexing function $\iota_G$, the graph nesting operator $\eta_{g_V,g_E,\mu_\Omega,\mu_E,\iota_G}^{\textbf{keep}}$ converts each subgraph in $G_C\in g_V(G_{i_c})$ (and $G_C\in g_E(G_{i_c})$) into a nested vertex (and nested edge) $\iota_G(G_C)$ and adds them in a newly-created nested vertex; vertices and edges in $G_{i_c}$ appearing neither in a nested vertex nor in a nested edge may be also returned if $\textbf{keep}$ is set to \texttt{true}. This operator returns the following nested graph:
	\[\begin{split}
	\eta&{}_{g_V,g_E,\mu_\Omega,\mu_E,\iota_G}^{\textbf{keep}}(G_{i_c})=G_{i_{\overline{c}}}=\\
	&=\Big\langle \{v\in \nu(i_c) | V(v)=\emptyset\wedge\textbf{keep} \}\cup \iota_G(g_V(G_{i_c})),\\
	&\qquad \{e\in \epsilon(i_c) | E(e)=\emptyset\wedge\textbf{keep} \}\cup \iota_G(g_E(G_{i_c}))\Big\rangle\\
	\end{split}\]
	where $\overline{c}=\max\{c|(i_c)\in\mathcal{V}\cup\mathcal{E}\}+1$. As a side effect of the graph nesting operation, the nesting graph database is updated using the nested graph classifier and user defined functions as follows:
	\begin{align*}
	\Big\langle&\VS\cup \iota_G(g_V(G_{i_c}))\cup\{(\overline{c},i)\},\quad \ES\cup \iota_G(g_E(G_{i_c})),\\
	& \lambda\oplus \bigoplus_{G_C\in g_E(G_{i_c})}\iota_G(G_C)\mapsto \mu_E(G_C),\\
	& \ell\oplus\bigoplus_{G_C\in g_E(G_{i_c})\cup g_V(G_{i_c})}\iota_G(G_C)\mapsto\texttt{fst}\;\mu_\Omega(G_C),\\
	& \omega\oplus\bigoplus_{G_C\in g_E(G_{i_c})\cup g_V(G_{i_c})}\iota_G(G_C)\mapsto\texttt{snd}\;\mu_\Omega(G_C),\\
	& \nu\oplus \bigoplus_{G_C\in g_E(G_{i_c})\cup g_V(G_{i_c})}\iota_G(G_C)\mapsto \mathcal{V}_C\\
	& \;\; \oplus dtl(i)_{\overline{c}}\mapsto \{v\in \nu(i_c) | V(v)=\emptyset\wedge\textbf{keep} \}\cup \iota_G(g_V(G_{i_c})),\\
	& \epsilon\oplus \bigoplus_{G_C\in g_E(G_{i_c})\cup g_V(G_{i_c})}\iota_G(G_C)\mapsto \mathcal{E}_C\\
	& \;\; \oplus dtl(i)_{\overline{c}}\mapsto\{e\in \epsilon(i_c) | E(e)=\emptyset\wedge\textbf{keep} \}\cup \iota_G(g_E(G_{i_c}))\Big\rangle\\
	\end{align*}
	where $(f\oplus g)(x)$ returns $g(x)$ if $x\in\textup{dom}(g)$ and $f(x)$ otherwise, and both $f$ and $g$ are finite domain functions. $a\mapsto b$ denotes a finite function, which domain contains only $a$.
\end{definition}

The following example describes the outcome of the graph nesting process.

\begin{example}[label=exNest]
	Figure \ref{fig:outputnested} provides the result of $\eta$ when the non-traversed vertices and edges are not preserved ($\textbf{keep}=\texttt{false}$) and where $V$ and $E$ are the ones represented in Figure \ref{fig:bibex2}. As showed by the former definition, the nesting operation updates the nested graph database by creating new nested vertices ($0_1,1_1,2_1$) and nested edges ($3_1,5_1,7_1,8_1$). Such nested components are contained within the returned nested graph $G_{11_1}$, which is represented as a nested vertex with the following members:
	\[\nu(11_1)=\{0_1,1_1,2_1\}\quad \epsilon(11_1)=\{3_1,5_1,7_1,8_1\}\]
	The nested graph database updated as a side effect of the graph nesting may be represented as follows:
	\[\begin{split}
	G'=\big\langle &\{0_0,1_0,\dots,5_0,11_0,0_1,1_1,2_1,11_1\},\\
	& \{6_0,\dots,10_0,3_1,5_1,7_1,8_1\}, \\
	& \lambda',\ell',\omega',\nu',\epsilon'\big\rangle\\
	\end{split}\]
	Let us now focus on the nested vertices and edges of $G_{11_1}$. As requested by the UDF functions, each resulting nested \textsc{Author}(2) preserves the original  vertices' tuple information, and its vertex members correspond to the \textsc{Paper}s authored by the corresponding \textsc{Author}(1). For easing the nested graph representation, we assume that each \textsc{Author}(2) has an associated  id $(1,i)$, which derives from a simple vertex with id $(0,i)$ in $G_{(0,11)}$. Therefore, vertex $0_1$ is represented as follows:
	\[\ell'(0_1)=[\mstr{Author}]\quad\nu'(0_1)=\{3_0\}\quad \epsilon'(0_1)=\emptyset\] \[\omega'(0_1)=\{\texttt{\textbf{name}}\colon\texttt{Abigail},\texttt{\textbf{surname}}\colon\texttt{Conner}\}\]
	
	Last, each resulting nested edge \textsc{coAuthorship} has a \mstr{coAuthorship} label, it has no tuple information and its vertex members correspond to the \textsc{Paper}s coauthored by source and target \textsc{Paper}.
	For easing the nested graph representation, we assume that each \textsc{coAuthorship} edge $a_1\to a'_1$ has an associated id $(\sum_{k=0}^{a+a'}k\;+\;a')_{1}$, which derives from the grouping references. Therefore, edge $0\to 2$ in Figure \ref{fig:outputnested} is represented as follows:
	\[\ell'(5_1)=[\mstr{coAuthorship}]\;\nu'(5_1)=\{(3_0)\}\;\epsilon'(0_1)=\emptyset\; \omega'(0_1)=\{\}\] 
\end{example}

 %Consequently, the following definition can be also given:

%\[\texttt{map}_{\ell,\xi,\phi\oplus [o.\nested_{c+1}\mapsto [\ONTA\mapsto \varphi(o.\nested_{c+1})]]}(\sigma_{s,\gamma_V}(\nested))\cup (\texttt{map}_{\ell,\xi,\phi\oplus [o.\nested_{c+1}\mapsto [\RELA\mapsto \varphi(o.\nested_{c+1})]]}(\sigma_{s',\gamma_E}(\nested)))\]

In particular, we can freely assume that our nested graph pattern matching semantics $s'$ acts as an UDF function, and hence associates to each graph cluster matched by $g_E$ a source and a target vertex. On the other hand, while the previous formal definition of the graph nesting operator provides a general definition matching with Algorithm \vref{alg:generalNesting}, the following  algorithms allow to match the class of graph nesting optimizable problems that is going to be defined in the next section.