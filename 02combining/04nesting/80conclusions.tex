\section{Conclusions}

To the best of our knowledge, this chapter proposed for the first time an algorihtm (THoSP) which adds structural aggregation to an input graph. The final outcome of this process is a nested graph, which contains vertices and edges that may contain subgraphs of the original input graph. Such result is obtained by jointly visiting two graph patterns, the vertex and the pattern summarization, respectively leading to the creation of nested vertices and nested edges. The reason why such algorithm outperforms equivalent implementations over graph, relational and document based competitors is twofold: first, while their query plans force one graph visit per pattern, our solution allows to visit such graph only once; last, by detaching the graph representation from the membership information in the \textit{query result} we can avoid the cost of performing an additional \texttt{GROUP BY} operation. This paper also provides a nested graph data model, allowing the definition of a generic graph nesting operator.

This solution was possible due to the assumptions derived from both GSM and GSQL, where it is showed that it is possible to refer any time to the elements that are going to be created later on within the computation, by simply deterministically knowing which the id belonging to the element that is going to be created. We already formulated such  assumption in our initial work on hypergraphs \cite{bergami2014}; this chapter proved the practical feasibility of this approach. Therefore, this chapter proves that the representation of nested graph may lead to the solution of current graph querying problems in a tractable way. Nevertheless, we believe that further studies will have to be done on the class of  GROQ problems, thus extending our work on THoSP.

This chapter walked in the footsteps of current (graph) database literature, where data operations are defined  at the single data structure level and not at the database level. As a consequence, the graph nesting operator must be represented as the creation of a new graph, requiring the update of the whole database as a side effect. On the other hand, all the operations that are performed on the GSM model via GSQL directly operate at the database level and provide the outcome of the nesting process as the final reference object. This chapter showed that the GSM data model provided a more clear and compact definition of the graph nesting operator (see Definition \vref{def:graphnesting}).

Last, this chapter (alongside with the former) outlined the definition of a possible nesting operator permitting the representation of $\nu_\cong$, which is the last operator required by the LAV/GAV data integration approach. Hereby, the definition of the graph nesting operator accomplishes our task of providing the full set of data integration operators via paNGRAm and GSQL. We believe that further studies will have to be done to extend the class of optimizable graph nesting patterns, and that a more general data model may provide a cleaner definition of the proposed graph nesting operator. 