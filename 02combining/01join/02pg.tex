

\section{Graph Data Model}\label{sec:datamodel}
Differently from the previously mentioned relational model, we now provide a data model allowing an explicit representation of tuples, which may appear more than once within a single set (e.g., vertex or edge set.). Still, we decide to embed the relational model on property graphs, so that the standard operators' properties from the relational model can be inherited.

We  model the
vertices' and edges' set 
as \textbf{multisets (of tuples)} $S$ of elements $s_i$, where  $s_i$ unequivocally identifies the $i$-th 
occurrence of a tuple $s$ in $S$. Each \textbf{tuple}\index{tuple} associates to each attribute a value:
it is a function $A\mapsto\mathcal{V}\cup\{\texttt{NULL}\}$ mapping each attribute in $A$ to  
either a  value in $\mathcal{V}$ or $\texttt{NULL}$ ($\varepsilon$ is the empty tuple).
We slightly change the previous property graph definitions in order to
ease the join definition between vertices and edges as later on required by the graph join:

\begin{definition}[Property Graph]\label{def:pg}
	A \textbf{property graph}\index{graph!property graph} is a tuple $G=(V,E,\Sigma_v,\Sigma_e,A_v,A_e,\lambda,\ell_v,\ell_e)$ 
	where \begin{alphalist}
		\item $V$ is a multiset of nodes,
		\item $E$ is a multiset of edges,
		\item $\Sigma_v$ is a set of node labels,
		\item $\Sigma_e$ is a set of edge labels,
		\item $A_v$ is a set of node attributes,
		\item $A_e$ is a set of edge attributes,
		\item $\lambda\colon E\to V\times V$ is a function assigning node pairs to edges,
		\item $\ell_v\colon V\to \mathcal{P}(\Sigma_v)$ is a function assigning a set of labels to nodes, and 
		\item $\ell_e\colon E\to \mathcal{P}(\Sigma_e)$ is a function assigning a set of labels to edges.
	\end{alphalist}
\end{definition} 

Given that the standard property graph model is unable to model graphs and databases in a similar representation, we must provide the following definition:

\begin{definition}[Graph Database]\label{def:gdb}
	A \textbf{graph database} is a collection of $n$ distinct property graphs $\{G_1,\dots,G_n\}$
	represented as a single 
	property graph $\mathcal{D}$ with $n$ distinct connected
	components. From now on we refer to each component simply 
	as \textbf{graph}. Each graph is identified by
	two functions: $\mathcal{V}\colon \{1,\dots,n\}\mapsto \mathcal{P}(V)$ determining the
	vertices $\mathcal{V}(i)$ of the $i$-th graph and $\mathcal{E}\colon \{1,\dots,n\}\mapsto \mathcal{P}(E)$ determining the edges $\mathcal{E}(i)$ of the $i$-th graph.
\end{definition}

\begin{example}[label=ex:onJoinDataModel]
	Two edges $e_i$ and $f_j$ come from two distinct graphs, respectively $G_a$ and $G_b$, within
	the same graph database $\mathcal{D}$. Edge $e_i$ connects vertex $u_h$ to $v_k$ ($\lambda(e_i)=(u_h,v_k)$), while $f_j$ connects $u'_h$
	to $v'_k$ ($\lambda(f_j)=(u'_h,v'_k)$).
	Such edges store only the following values:
	\[e_i(\textsc{Time})=12\textup{:}04,\quad f_j(\textsc{Day})=Mon\]
	and have the following labels:
	\[\ell_e(e_i)=\Set{\textup{Follow}},\quad\ell_e(f_j)=\Set{\textup{FriendOf}}\]
\end{example}

For the multiset $\theta$-join, we need a function $\oplus$ combining two tuples for the
relational join operator over multisets, where $r_i\oplus t_j$ is a valid
multiset element $(r\oplus s)_{i\oplus j}$ and $i\oplus j$ maps 
each integer pair $(i,j)$ to a single number. If we define $\oplus$ as a linear function (that is for each function 
$H$, $H(e_i\oplus f_j)=H(e_i)\oplus H(f_j)$),
the $\theta$-join also induces the definition of $\ell_v$, $\ell_e$ and $\lambda$ for
the joined tuples. As a consequence, $\oplus$ must be 
overloaded for each possible expected output from $H$. Such function is defined as follows:


\begin{definition}[Concatenation]\label{def:concatenation}
	\index{concatenation}
	$\oplus:A\times A\mapsto A$ is a {{lazy evaluated}} \textbf{concatenation}\index{concatenation|see {$\oplus$}}\index{$\oplus$!higher order} function between two 
	operands of type $A$ returning an element of the same type, $A$. The concatenation function is a linear function such that, given any function $H$ with $dom(H)=A$, $H(u\oplus v)=H(u)\oplus H(v)$. $\oplus$ is defined for the following $A$-s:
	\begin{itemize}
		\item \textbf{sets}: it performs the union of the two sets:\qquad
		$S\oplus S'\overset{def}{=}S \cup S'$
		\item \textbf{integers}: it returns the dovetail number associating to each pair of
		integers an unique integer:
		$i\oplus j\overset{def}{=}\sum_{k=0}^{i+j}k+i$
		%%		
		%%		
		%%		\item \textbf{pair of integers}: given two natural numbers $a$ and $b$, $a\oplus b$ is the dovetailing
		%%		function \cite{odi} uniquely mapping a pair of natural numberss into a single natural number:
		%%		\[a\oplus b = \sfrac{(a+b)(a+b+1)}{2}+a\]
		%%		
		\item \textbf{functions}: given a function $f:A\mapsto B$ and $g:C\mapsto D$, $f\oplus g$ is the overriding\index{overriding} of $f$ by $g$ returning $g(x)$ if $x\in \dom(g)$, and $f(x)$ if $x\in \dom(f)$. \texttt{NULL} is returned otherwise.
		Such function concatenation are used in joins when $\forall x\in A\cap C. f(x)=g(x)$.\index{function overriding|see {$\oplus$, higher order}}
		
		\item \textbf{pairs}: given two pairs $(u,v)$ and $(u',v')$, then
		the pair concatenation is defined as the pairwise {{concatenation}} of each element, that is $(u,v)\oplus (u',v')\overset{def}{=}(u\oplus u',v\oplus v')$. 
		Elements belonging to multisets are represented as pairs of
		elements and integers, and hence $s_i\oplus t_j\overset{def}{=}(s\oplus t)_{i\oplus j}$.
		
	\end{itemize}
\end{definition}

After providing the definition of the concatenation function, we can provide the graph join definition as follows:

\begin{definition}[$\theta$-Join]\label{def:thetajoin}
	Given two (multiset) tables $R$ and $S$ over a set
	of attributes $A_1$ and $A_2$, the \textbf{$\theta$-join} $R\bowtie_\theta S$ \cite{atzeniEN,atzeniIT} is defined as follows:
	\begin{equation*}
	R\bowtie_\theta S=\{r_i\oplus s_j\mid r_i\in R, s_j\in S,\theta(r_i,s_j), (r_i\oplus s_j)(A_1)=r_i, (r_i\oplus s_j)(A_2)=s_j\}
	\end{equation*}
	where $(t\oplus t')(A_i)$ denotes the projection of the tuple $t\oplus t'$ over $A_i$.
	If $\theta$ is the always true predicate, $\theta$ {{can}} be omitted and, when also $A_1\cap A_2=\emptyset$, we have a cartesian product.
\end{definition} 

%We want to define a $\theta$-join for vertices' and edges' multisets of tuples. If we
%extend the $\oplus$ operator for the elements of a multiset $S$, $s_i$ and $t_j$, such that
%$s_i\oplus t_j$ is an element $(s\oplus t)_{i\oplus j}$ where $i\oplus j$ is an
%unique integer for the pair of integers $(i,j)$, then we have that the multiset 
%$\theta$-join returns a multiset
%since the returned elements are in the form $(s\oplus t)_{i\oplus j}$.
%However, vertices have also a labelling function $\ell_v$ mapping each vertex into a set of labels,
%and edges have both a labelling function $\ell_e$ (mapping each vertex into a set of labels)
%and a $\lambda$ function (mapping each tuple into a pair of vertices, the source and the 
%destination of each edge) \cite{Neo4jAlg}:




\begin{example}[continues=ex:onJoinDataModel]
Suppose now that the edge $e_i\oplus f_j$ comes from
	a graph join where edges from $G_a$ are joined to the ones in $G_b$ in a resulting graph,
	where also vertices $u_h\oplus u'_h$
	and $v_k\oplus v'_k$ appear. So:
	\[(e_i\oplus f_j)(\textsc{Time})=12\textup{:}04,\quad (e_i\oplus f_j)(\textsc{Day})=Mon\]
	By $\oplus$'s linearity, we have that the labels are merged:

	\[\ell_e(e_i\oplus f_j)=\ell_e(e_i)\oplus\ell_e(f_j)=\{\textup{Follow}\}\oplus\{\textup{FriendOf}\}=\{\textup{Follow},\textup{FriendOf}\}\]
	And the result's vertices are updated accordingly:
	\[\lambda(e_i\oplus f_j) = \lambda(e_i)\oplus\lambda(f_j)=(u_h,v_k)\oplus(u'_u,v'_k)=(u_h\oplus u'_h,v_k\oplus v'_k)\]
\end{example}

Since all the relevant informations are stored in the graph database, we represent 
the graph as the set of the minimum information required for the join operation.

\begin{definition}[Graph]
	The $i$-th \textbf{graph} of a graph da\-tabase $\mathcal{D}$ 
	is a tuple
	$G_i=(\mathcal{V}(i),\mathcal{E}(i),A_v^i,A_e^i)$, where $\mathcal{V}(i)$ is a multiset of vertices and 
	$\mathcal{E}(i)$ is a multiset of edges. Furthermore, $A_v^i$ is a set of attributes 
	$a\in A_v^i$ s.t. there is at least one vertex $v_j\in \mathcal{V}(i)$ having $v_i(a)\neq\texttt{NULL}$; $A_e^i$ is a set of attributes $a'\in A_e^i$ s.t. there
	is at least one edge $e_k\in \mathcal{E}(i)$ having $e_k(a')\neq\texttt{NULL}$.
\end{definition}

%Our graph join algorithm relies over a specific graph data model, the property graph 
%model, which is the most general graph representation available.  It 
%allows multiple vertices having the same values and multiple edges between two nodes. 
%Since graph databases (e.g. Neo4J  \cite{Neo4jMan}) allow multiple labels, labels are 
%defined as string sets (e.g. \{User\}).
%
%
%Such algorithm has to produce a graph that contains only the vertices' and the 
%edges' identifiers, in order to produce the results as fast as possible. Such identifiers must
%reconstruct the whole graph information, such as \begin{ienum}
%	\item the attribute-values association over each vertex and edge, and 
%	\item the vertices' and edges' labels, alongside with the edges' source and destination
%	vertices ($\lambda$ function, \cite{Neo4jAlg}).
%\end{ienum} 
%

%
%Last, the graph join – as the relational one – must handle only the tuples' values and
%ignore the graph database schema (e.g. the vertices' and edges' labels). In order
%to do so we decouple $\lambda$ and labels from our (property) graph and store them in the graph
%database containing them. 
%



%
%\begin{definition}[Graph Database]
%	A graph database $\mathcal{D}$ is defined as a collection of graphs $\Set{G_i}_{i\leq n}$ alongside with
%	an ``enrichment'' $\Phi=(\Sigma_v,\Sigma_e,\lambda,\ell_v,\ell_e)$:
%	\begin{enumerate*}[label=\textit{(\roman*)}]
%		\item $\Sigma_v$ is a set of vertices' labels
%		and \item $\Sigma_e$ is a set of edges' labels;
%		furthermore, 
%		\item for each graph $G_i\in \mathcal{D}$, $\lambda$
%		is a function $\bigcup_iE_i\times\mathbb{N}\mapsto V_i\times V_i$ mapping each 
%		pair\footnote{$e_j$ is an edge belonging to the edges' multiset 
%			$E_i$ from graph $G_i\in \mathcal D$.
%			Similarly, a pair $(v_j,i)$ refers to a vertex $v_j$ 
%			belonging to a vertex multiset $V_i$ belonging to a graph $G_i\in \mathcal D$.
%			\label{fnm1}} $(e_j,i)$
%		to pair of vertices (source and destination) in $G_i$. \item 
%		$\ell_v$ is a function $\bigcup_iV_i\times\mathbb{N}\mapsto \Sigma_v$ mapping each pair\footref{fnm1} $(v_j,i)$
%		to its label. \item  
%		$\ell_v$ is a function $\bigcup_iE_i\times\mathbb{N}\mapsto \Sigma_E$ mapping each pair $(e_j,i)$ to its label.
%		When the $i$-th graph is clear from context, then the previous pairs $(v_j,i)$ and $(e_j,i)$ are
%		represented as $v_j$ and  $e_j$.
%	\end{enumerate*}
%\end{definition}
%
%We now have to show how such information can be reconstructed when a result graph 
%is obtained through the composition of tuples (vertices and edges) coming from two distinct
%operators, as in the case of graph join. 
%
%
%Due to the properties of $\oplus$ (see Appendix), we can reconstruct such information:
%\begin{example}
%	Given two edges $e_i$ and $f_j$ coming from two distinct graphs $G_a$ and $G_b$ from
%	a same graph database linking some vertices $\lambda(e_i,a)=(u,v)$ and $\lambda(f_j,b)=(u',v')$,
%	those are successfully joined as $e_i\oplus f_j$ if both $u\oplus u'$ and $v\oplus v'$ appear in the 
%	graph join's vertex set. Given:
%	\[e_i(\textsc{Time})=12\textup{:}04,\quad\ell_e(e_i,a)=\Set{\textup{Follow}}\]
%	\[f_j(\textsc{Day})=Mon,\quad\ell_e(f_j,b)=\Set{\textup{FriendOf}}\]
%	then, by $\oplus$ definition we have that the joined edge links the
%	joined vertices, $\lambda(e_i\oplus f_j)=(u\oplus u',v\oplus v')$, 
%	its label is the union of the two edges' label set, $\ell_e(e_i\oplus f_j)=\Set{\textup{Follow},\textup{FriendOf}}$, and also their tuples are 
%	joined, hence $(e_i\oplus f_j)(\textsc{Time})=12\textup{:}04$ and 
%	$(e_i\oplus f_j)(\textsc{Day})=Mon$.
%\end{example}


%%Prior to the definition of our join operator (that is going to be defined in the next Section), we have to
%%provide the data model on which our operator is based on. We choose
%%to use the property graph model as our baseline because it is the most
%%general graph representation available \cite{Neo4jAlg}. 
%%We intend a graph
%%database as a collection of graphs, as in Figure \ref{fig:exdb}. 
%%
%%
%%
%%Since most of current graph databases allow to insert multiple vertices
%%having the same values and multiple edges between two nodes, we
%%define both 

%%Our graph data model has to 
%%induce a graph-equality that is label-independent,
%%as in relational algebra where the relations (tables) equivalence does not
%%consider the name of the output relation.
%%This requirement is met by defining a property graph as a value store for both 
%%vertices and edges:


%Labels are defined as sets of string (e.g. \{User\}): some graph databases (e.g. Neo4J  \cite{Neo4jMan}) allow
%multiple labels.
%%Each graph stores no information
%%about the edges' sources and destinations, that are stored through a $\lambda$
%%function \cite{Neo4jAlg}, here at the graph database level. Similar considerations follows
%%for the graphs' labels.


